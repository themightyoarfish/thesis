\chapter{Conclusion}

This thesis presented an attempt at retrieving the location of a photograph with
the help of image processing. After introducing the basic geometry underlying
world-to-image mapping and the geometry relating two views of a scene, a
computational approach developed by \citet{bae2010} has been presented.
Structure-from-motion is used to compute the motion between two captures of the
same scene and infer the original viewpoint from the reference photo and two
images of the current scene.  The theoretical and practical problems have been
highlighted, including such scenes which do not allow determining the relative
pose of cameras viewing them and how to acquire matching points in two images.
Algorithms to solve the pose recovery problem have been introduced.

Evaluation on two real scenes revealed that only two elements---the necessary
camera rotation and the closeness to the goal---show promising estimates with
this approach. The direction of necessary translation could not be recovered
with this method.  Since this
aspect is not further documented in Bae et. al's work, it warrants further
investigation.

An implementation of most of the ideas as a mobile application has been
prototyped and described, being the first of its kind despite not being as
functional as intended. The application requires the user to load an image,
instructs them two capture two images of the scene and then visualises an
estimate for the necessary motion with two arrows, one for the sensor plane and
one for the optical axis. Further assistance is provided by overlaying the edges
of the original image over the current camera picture, making the application
useful despite the failure of the arrow visualisation.  When the final image is
captured, the app lets the user review the rephoto. A slider is used to mask the
old over the new image and the change of the scene can be dynamically
visualised. Rephotos are saved to the gallery to be viewed at a later time.

It must be concluded that the original objective could not be achieved in its
entirety and
the application is currently not more useful than existing approaches. With an edge
overlay, it can still help a user in capturing rephotographs, although they
still need to determine the necessary motion on their own. There is hope
however, that a solution to the translation problem can be found and
implementedin the future, increasing the capabilities of the app.
