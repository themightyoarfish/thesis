\chapter{Conclusion}

This thesis gave an overview of how to retrieve the location of a photograph
with help of image processing. After introducing the basic geometry underlying
world-to-image mapping and the geometry relating two views of a scene, a
computational approach developed by Bae, Durand and Argawala has been presented.
The theoretical and practical problems have been
highlighted, including scenes which do not allow determining the relative pose
of cameras and how to acquire matching points in two images. Algorithms to solve
the pose recovery problem have been introduced.

Evaluation revealed that only certain elements---the necessary
camera rotation and the closeness to the goal---show promising estimates. The
direction of necessary translation could not be recovered with this method.
Since this aspect is not further documented in Bae et. al's work, it warrants
furhter investigation.

An implementation of most of the ideas as a mobile application has been prototyped and
described, being the first of its kind despite not being as functional has
intended. Several simplifications have been made and the 

It must be concluded that the original objective could not be achieved and
the application is not more useful than existing approaches. With an edge
overlay, it can still help a user in capturing rephotographs, although they
still need to determine the necessary motion on their own.
