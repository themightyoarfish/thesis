\chapter{Camera Geometry}

This chapter will introduce the geometry of one and two views (the epipolar geometry) and
how it can be used to recover relative camera position from two images of the
same scene. It largely follows \citep[chapters 6,7]{h&z2004}.

\section{Camera Models}

Given a camera $C$ whose center of projection is the origin and a point
$\mbf{X}$ in camera-centric coordinates, the central projection of
$\mbf{X}$ onto $C$'s image plane is depicted in a side-view in
\autoref{pinhole}.  The image plane is virtually moved to the front of the
camera, otherwise the image would be mirrored at the principal point as in real
cameras.  Let $f$ be the focal length, which is the distance of the image plane
to the optical centre.  If $\mbf{X} = (X,Y,Z)$, then $x=\left(f \frac{X}{Z},
f \frac{Y}{Z}, f\right)$ by use of the intercept theorem, with $\left(fX,
fY\right)^T$ being the coordinates in the image plane.

\begin{figure}[h]
   {\centering      
      \begin{tikzpicture}[every node/.style={outer sep=0,inner sep=0},
   label distance=5pt]
   \usetikzlibrary{calc}
   \fill (0,0) circle (2pt) node[label=below:{$C$}] (C) {};
   \draw[->,name path=y-axis] (C) -- ++(0,4cm);
   \draw[->,name path=x-axis] (C) -- ++(8cm,0);

   \fill (7cm,3cm) circle (2pt) node[label=above right:{$\mathbf{X}$}] (X) {};

   \draw[name path=image plane] (3cm,1.7cm) -- ++(0,-2*1.7cm);

   \draw[|-|] (0,-2cm) -- (3cm,-2cm) node[midway,below=1em] (f) {$f$};

   \draw[name path=ray] (C) -- (X);
   \fill[name intersections={of=ray and image plane}]
   (intersection-1) circle (2pt) coordinate (c);
   \node[label=above left:{$x$}] (x) at (c) {};

   \draw[dashed] (0,3cm) node[left=1em] {$Y$} -- (X) -- (7cm,0)
   node[below=1em]
   {$Z$};

   % \draw[|-|] (0,-3cm) -- (7cm,-3cm) node[midway,below=1em] (Z) {$Z$};

   \draw[|-|,name intersections={of=x-axis and image plane}] (x) ++(1em,0) --
   ($(intersection-1) + (1em,0)$) node [right=1em,midway] {$f \cdot
   \frac{Y}{Z}$};

\end{tikzpicture}

      \caption{Central projection for a pinhole camera}
   \label{pinhole}}
\end{figure}


When representing the points as homogeneous quantities, the central projection
can be expressed by a matrix multiplication. 
\marginnote{Homogeneous vectors are
the elements of projective geometry. They can be obtained from cartesian
coordinates by appending a 1-element. All projective entities which differ only
by a scalar factor are equivalent, one writes $\mbf{x} \sim \mbf{y}$ if
$\mbf{x} = \lambda\mbf{y}, \lambda \neq 0$. This has the added effect that
points at infinity can be represented by vectors whose last coordinate is zero.}
This can be written with homogeneous coordinates as
\begin{IEEEeqnarray*}{rClCl}
   \left(
      \begin{array}{c}
         f\frac{X}{Z} \\ f\frac{Y}{Z}\\ 1
      \end{array}
   \right) & \sim &
      \left(
         \begin{array}{c}
            fX \\ fY \\ Z
         \end{array}
      \right) & = & \underbrace{\addstackgap[6pt]{\left(
         \begin{array}{cccc}
            f & 0 & 0 & 0 \\
            0 & f & 0 & 0 \\
            0 & 0 & 1 & 0
         \end{array}
\right)}}_{\text{Projection Matrix of $C$}} \left(\begin{array}{c} X \\ Y \\ Z \\ 1 \end{array}\right)
\end{IEEEeqnarray*}

or in short.
\begin{equation}
   \mbf{x} = P\mbf{X}
\end{equation}

The above situation is a special case wherein the camera centre $C$ defines the
origin and the optical and image axes are the coordinate axes. Thus the rotation
and translation of the camera relative to this coordinate system is zero. More
generally, there might be a world coordinate frame with different origin 
and different axes, so that a coordinate transform must be applied to $\mbf{X}$
before the projection. 

Let $R \in \mathbb{R}^{3\times3}$ be a rotation matrix
giving the camera's rotation relative to the world frame and $t \in
\mathbb{R}^{3\times1}$ its translation such that

\begin{equation}
   \mbf{X}_{\text{cam}} = R \mbf{X}_{\text{world}} + t
\end{equation}

Then the projection of a point $\mbf{X}$
in world coordinates onto the image plane becomes

\begin{IEEEeqnarray*}{rCl}
   \mbf{x} = K\cdot\left[R \mid t\right] \mbf{X}
\end{IEEEeqnarray*}


Real cameras are not ideal pinhole cameras. Furthermore, it is useful to have
the dimension of all values be pixel units.
A camera has five intrinsic parameters and can be written in matrix form as
\begin{IEEEeqnarray*}{rCl}
   K & = & \left(
   \begin{array}{ccc}
      f_x & s     & c_x \\
      0   & f_y   & c_y \\
      0   &       & 1
   \end{array}
\right)
\end{IEEEeqnarray*}
where $f_x$ and $f_y$ are the focal lengths in $x$- and $y$-directions expressed
in pixel units ($f_x$ and $f_y$ are not necessarily identical, if the sensor has
non-square pixels), $s$ the sensor skew (the pixels may not be rectangular;
their edges may not be perpendicular) which is usually zero, and the coordinates
of the principal point $(c_x,c_y)$ with respect to the origin of the image plane
which usually placed at the upper left corner. The principal point is the
intersection of the image plane with the optical axis.

The intrinsic camera parameters assembled in $K$ are therefore essential to
relate world points to image points which will be important for this
application. In theory, these parameters could be obtained from the camera's
vendor who knows the precise manufacturing specification. In practice, only the
focal lengths $f_x, f_y$ are known, in most cases only one with the assumption
of square pixels. Usually, the principal
point is assumed to be at the sensor centre and the pixels are assumed to be
rectangular. In practice however, there are variances introduced by different
causes such as imprecise manufacturing or physical impacts which may decentre
the lens such that the principal point is no longer at the centre. 

A further compilcation is introduces by the camera lens which will often have a
non-negligible distortion, most prominently radial distorion as depicted in
\autoref{distortion}. It can be modeled by the application of a 
distortion factor to the ideal undistorted image coordinates $(\tilde{x}, \tilde{y})$.

\begin{figure}
   {\centering      
      \begin{tikzpicture}[every to/.style={color=red}]
   \newcommand{\yshift}{6cm}
   \draw (0,0) rectangle (4,3);
   \begin{scope}[xshift=2cm,yshift=1.5cm]
      \draw (-1.5,-1) to[bend right=8] (1.5,-1) to[bend right=8] (1.5,1) to[bend
      right=8] (-1.5,1) to[bend right=8] (-1.5,-1);
   \end{scope}
   \begin{scope}[xshift=\yshift]
      \draw (0,0) rectangle (4,3);
      \begin{scope}[xshift=2cm,yshift=1.5cm]
         \draw (-1.5,-1) to (1.5,-1) to (1.5,1) to (-1.5,1) to (-1.5,-1);
      \end{scope}
   \end{scope}
\draw[-{Latex},thick,color=RoyalBlue,text=black] (4,1.5) ++(1mm,0) -- node[below,midway,font=\footnotesize]
   {correction} ($(\yshift,1.5) - (1mm,0)$);
\end{tikzpicture}

   \caption{Radially distorted image on the left, the corrected image on the
   right.}
   \label{distortion}}
\end{figure}


\begin{equation}
   \begin{pmatrix}
      x_d \\ y_d
   \end{pmatrix} = L(r)\begin{pmatrix}
      \tilde{x} \\ \tilde{y}
   \end{pmatrix}
\end{equation}

where $L$ is a nonlinear function of the distance $r$ from the distortion
center---usually coincident with the principal point. The function can be
approximated as an exponential with a Taylor expansion
\begin{equation}
   L(r) = 1 + \sum\limits_{i=1}^k \kappa_i r^i
\end{equation}
for a given $k$ \citep[see][chapter 7.4]{h&z2004}. The intrinsic camera
parameters which consist in the entries of $K$ and distortion coefficients
$\kappa_i$ must be determined in order to accurately relate world coordinates to
image coordinates. They can be found by calibrating the camera. Different
methods exist \citep[e.g][]{zhang2000} but will not be examined here.

\section{Epipolar Geometry}

Epipolar geometry is the geometry which relates the image points in two views of
the same scene. \autoref{epipolar} shows the basic setup. 

We consider a scene viewed by two cameras with optical centres $c_1$ and $c_2$,
world points $\mbf{X}_i$, where the subscript denotes the coordinate
frame---the first camera, arbitrarily chosen to be the left one, or the second
camera---and image points $\mbf{x}_i$ which are the projections of $\mbf{X}_i$
onto the image planes and thus correspond to the same world point. 
It is obvious that the following relation holds
\begin{equation}
   \lambda_i\mbf{x}_i = \mbf{X}_i
\end{equation}
that is, the world point lies on a ray through the optical centre and the image
point. 

\begin{figure}[h]
   {\centering      
      \usetikzlibrary{intersections, calc, arrows.meta}
\begin{tikzpicture}[every node/.style={outer sep=0,inner sep=0},
   label distance=5pt]
   \def\baseline{10cm}
   \draw[fill] (0,0) circle (2pt) node[label=below left:{$c_1$}] (c1) {};
   \draw[fill] (\baseline,0) circle (2pt) node[label=below left:{$c_2$}] (c2) {};

   \newcommand{\sensorplane}[1]{
      \path[name path=#1,draw] (1cm,0.5cm) -- (1cm,2.5cm) -- (3.5cm,1cm) -- (3.5cm,-1cm) -- cycle
   }
   \begin{scope}
      \sensorplane{plane1};
   \end{scope}

   \begin{scope}[yscale=1,xscale=-1,xshift=-\baseline]
      \sensorplane{plane2};
   \end{scope}

   \fill (0.5*\baseline,3cm) circle (2pt) node[label=above:{$\mathbf{X}_1$}]
   (X1) {};

   \draw (c1) -- ++(3cm,0) [fill] circle (2pt) node[label=above:{$e_1$}] (e1)
   {} coordinate (c);
   \draw[dotted] (c) -- (3.5cm,0) coordinate (c);
   \draw[solid] (c) -- (\baseline-3.5cm,0) coordinate (c);
   \draw[dotted] (c) -- (\baseline-3cm,0) coordinate (c);
   \draw[fill] (c) circle (2pt) node[label=above:{$e_2$}] (e2) {} -- (c2);

   % lines from camera center to x
   \path[name path=c1 to X1] (c1) -- (X1) coordinate[pos=.3] (c);
   \draw[solid,fill] (c1) -- (c) coordinate (x1) circle (2pt);
   \node[label=above:{$x_1$}] (x1) at (c) {};
   \draw[dotted,name intersections={of=c1 to X1 and plane1}] (c) --
   (intersection-2) coordinate (c);
   \draw[solid] (c) -- (X1);

   \path[name path=c2 to X1] (c2) -- (X1) coordinate[pos=.3] (c);
   \draw[solid,fill] (c2) -- (c) coordinate (x2) circle (2pt);
   \node[label=above:{$x_2$}] (x2) at (c) {};
   \draw[dotted,name intersections={of=c2 to X1 and plane2}] (c) --
   (intersection-2) coordinate (c);
   \draw[solid] (c) -- (X1);

   % draw epilines
   \draw[RoyalBlue,text=black] (e2) -- (x2) node[midway,label=above:{$l_2$}] (label-l2) {};

   \draw[RoyalBlue,text=black] (e1) -- (x1) node[midway,label=above:{$l_1$}] (label-l1) {};

   % draw relative rotation and translation
   \draw[-{Latex}] ($(c1) + (2em,-2em)$) to[bend right=50] node[below=1em, midway]
   {$R$, $T$} ($(c2) + (-2em,-2em)$);

\end{tikzpicture}

      \caption{Basic epipolar geometry}
   \label{epipolar}}
\end{figure}

Given camera centers $c_1$ and $c_2$ with their respective image planes  and a point $\mbf{X}_1$ in world
coordinates, it is projected onto
