\chapter{Outlook}
\section{Bugs}
\section{Future Extensions \& Improvements}

The focus of this work was the theoretical exploration and evaluation of Bae et.
al's ansatz to computational rephotography, not on the design of release-ready
application. As such, some functionality is missing and the app is not
extensively tested. Notable shortcomings are
\begin{enumerate}
   \item In the gallery inside the app, the final captures of each rephoto are
      shown as thumbnails. Should multiple rephotos be based on the same
      original, they will be indistinguishable. The thumbnails should be made
      more meaningful.

   \item Deletion of rephotos works via the user's gallery, where the rephoto
      album created by the app is visible. Deletion will leave invalid entries
      in the database which should be cleaned up by the app. More reasonably
      perhaps, deletion should be offered inside the app.

   % \item Since rephotos are saved as tuples of URLs it may happen that the user
   %    takes multiple rephotos with the same original. Currently, both will be
   %    displayed in the gallery inside the app, but the \code{MainViewController}
   %    picks the first record for which the URL of the selected image matches so
   %    that only the first rephoto can be reviewed. Additional logic would

   \item The upload of rephotos is experimentally hardcoded to work with a test
      server and is a blocking request.

   \item To enhance user experience, a choice between different overlays should
      be offered, where at this time only an edge overlay is implemented.

   \item All frames are matched with the first frame so it would suffice to
      compute the keypoints once for it and reuse them. Currently, the keypoints
      are recomputed for every \code{PoseDifferenceEstimationOperation}.

   \item The application requires iOS 8, but actually has few dependencies on
      it. The user interface employs \code{UIVisualEfffectView}s which were
      previously unavailable. It should be fairly straighforward to create
      versions of the UI without these elements.

   \item The rephotography makes use of the simple
      \code{UIImagePickerController} which is presented modally. This makes the
      flow of the app less seamless. For instance, when the
      \code{PhotoChooserController} is presented, and the user loads an image,
      the image picker will pop up, but upon selection, one is not immediately
      shown the camera, but is briefly shown the photo chooser again before the
      rephoto starts. Furthermore, there is no possibility to modify the camera
      stream which would be necessary to implement the image warping used by
      \citet{bae2010}. Also, images are captured by a timer, it would be more
      intuitive to attempt to process frames as they come and simply drop them
      if no CPU time is available. Both could be accomplished by rebasing the
      app on the much more customisable \code{AVFoundation} framework which
      gives direct access to the device's camera. For this, OpenCV wrappers
      exist already which would simplify processing the frames.
\end{enumerate}

