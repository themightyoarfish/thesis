\chapter{Outlook}\label{ch:outlook}
\section{Future Work}


\section{Usability And Performance}

The focus of this work was the theoretical exploration and evaluation of Bae et.
al's ansatz to computational rephotography, not on the design of release-ready
application. As such, some functionality is missing and the app is not
extensively tested. Notable shortcomings are
\begin{enumerate}
   \item In the gallery inside the app, the final captures of each rephoto are
      shown as thumbnails. Should multiple rephotos be based on the same
      original, they will be indistinguishable. The thumbnails should be made
      more meaningful.

   \item Deletion of rephotos works via the user's gallery, where the rephoto
      album created by the app is visible. Deletion will leave invalid entries
      in the database which should be cleaned up by the app. More reasonably
      perhaps, deletion should be offered inside the app.

      % \item Since rephotos are saved as tuples of URLs it may happen that the user
      %    takes multiple rephotos with the same original. Currently, both will be
      %    displayed in the gallery inside the app, but the \code{MainViewController}
      %    picks the first record for which the URL of the selected image matches so
      %    that only the first rephoto can be reviewed. Additional logic would

   \item The upload of rephotos is experimentally hardcoded to work with a test
      server and is a blocking request.

   \item To enhance user experience, a choice between different overlays should
      be offered, where at this time only an edge overlay is implemented.

   \item All frames are matched with the first frame so it would suffice to
      compute the keypoints once for it and reuse them. Currently, the keypoints
      are recomputed for every \code{PoseDifferenceEstimationOperation}.

   \item The application requires iOS 8, but actually has few dependencies on
      it. The user interface employs \code{UIVisualEfffectView}s which were
      previously unavailable. It should be fairly straighforward to create
      versions of the UI without these elements.

   \item The calibration data cannot currently be loaded from a user-supplied
      file. Ideally, since not all deviced will have the exact same camera
      chracteristics, each user would have to individually calibrate their
      camera.

   \item The rephotography makes use of the simple
      \code{UIImagePickerController} which is presented modally. This makes the
      flow of the app less seamless. For instance, when the
      \code{PhotoChooserController} is presented and the user loads an image,
      the image picker will pop up, but upon image selection, one is not
      immediately shown the camera, but is briefly shown the photo chooser again
      before the rephoto starts with yet another \code{UIImagePickerController}.
      Furthermore, the possibility to modify the camera stream is limited to
      affine transformations, which precludes projective transformation. This
      would be necessary to implement the image warping used by \citet{bae2010}.
      Also, images are captured by a timer, it would be more intuitive to
      attempt to process frames as they come and simply drop them if no CPU time
      is available which would also scale more easily to other devices. Both
      could be accomplished by rebasing the app on the much more customisable
      \code{AVFoundation} framework which gives direct access to the device's
      camera. For this, OpenCV wrappers exist already which would simplify
      processing the frames.  

   \item The application has only been tested on a single device. While it
      should work on iPhones as well, this could not be tested and some tuning
      may be necessary.

   \item The application creates a photo album for the rephotos with the name
      \emph{Rephotos}, without checking if it already exists.

\end{enumerate}


Some features from Bae et. al's work have been simplified away, so that
currently, no really historic photograph can be used. The app should therefore
be extended with another view controller presented after first and second frames
have been taken, in order for the user to label correspondences. Furthermore, a
higher rate of processing could be achieved to also use fast feature tracking
between robust pose estimations where the AKAZE detector is run. Bae et al. use
Lucas-Kanade-Tomasi tracking for this \citep{lucas1981,tomasi1991}, an OpenCV
implementation is already available. During automatic capture, it can be
observed that the found features often change abruptly between successive
frames. This and the fact that the number of found features changes strongly
results in the scale estimation jumping rapidly. A feature tracking may increase
the stability as the already-found points are found again, instead of radically
different ones.

\subsection{Rephoto Postprocessing}

It is probable that despite all assistance, the user will not manage to
perfectly recreate the original photgraph.  In order to better register the
original and final images, the latter can be warped according to the last pose
difference estimate. Bae et. al use an infinite homography. The infinite
homography is the one between two views induced by the plane at infinity, i.e.
the set of all projective points $(x,y,z,0)$ and maps vanishing points in one
view onto vanishing points in the other. Details can be found in \citep[ch. 13.4]{h&z2004}.
Given the calibration matrices of both views and their relative
rotation, the homography can be computed and one image can be warped with it
by remapping the pixels.

\subsection{Other Visualisation Options}\label{subsec:alternative_visualisation}

\citet{bae2010} explored different ways of visualising the necessary motion.
A side-by-side view or a linear blend were less helpful than the arrow
visualisation, but one could image to offer evaluate different kinds, such as
a three-dimensional arrow which shows both parts of the necessary translation
or. A more sophisticated idea is to project an approximation of the correct
location into the camera picture, but this would require a more detailed
reconstruction of the scene.
