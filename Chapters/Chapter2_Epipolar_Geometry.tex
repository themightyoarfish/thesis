\chapter{Camera Geometry}

This chapter will introduce the geometry of image projection, largely following
\citep[chapters 6,7]{h&z2004} and the geometry of two views (the epipolar
geometry, \citet[ch. 5.1]{ma2003}) and
how it can be used to recover relative camera position from two images of the
same scene. 

\section{Camera Models}

The camera model used is the ideal pinhole camera model which postulates several
idealised assumptions.
\begin{enumerate}
   \item The aperture size is infinitely small
   \item There are no lens effects (\emph{thin lens} assumption)
   \item The angle of view is arbitrarily large
   \item All world points projected onto the image plane are in focus, owing to
      the small aperture
\end{enumerate}

\marginnote{Homogeneous vectors are
the elements of projective geometry. They can be obtained from cartesian
coordinates by appending a 1-element. All projective entities which differ only
by a scalar factor are equivalent, one writes $\mbf{x} \sim \mbf{y}$ if
$\mbf{x} = \lambda\mbf{y}, \lambda \neq 0$. This has the added effect that
points at infinity can be represented by vectors whose last coordinate is zero.}

Given a camera $C$ whose center of projection is the origin and a point
$\mbf{X}$ in camera-centric coordinates, the central projection of
$\mbf{X}$ onto $C$'s image plane is depicted in a side-view in
\autoref{pinhole}.  The image plane is virtually moved to the front of the
camera, otherwise the image would be mirrored at the principal point as in real
cameras.  Let $f$ be the focal length, which is the distance of the image plane
to the optical centre.  If $\mbf{X} = (X,Y,Z)$, then $x=\left(f \frac{X}{Z},
f \frac{Y}{Z}, f\right)$ by use of the intercept theorem.%, with $\left(fX, fY\right)^T$ being the coordinates in the image plane.


\begin{figure}[h]
   {\centering      
      \begin{tikzpicture}[every node/.style={outer sep=0,inner sep=0},
   label distance=5pt]
   \usetikzlibrary{calc}
   \fill (0,0) circle (2pt) node[label=below:{$C$}] (C) {};
   \draw[->,name path=y-axis] (C) -- ++(0,4cm);
   \draw[->,name path=x-axis] (C) -- ++(8cm,0);

   \fill (7cm,3cm) circle (2pt) node[label=above right:{$\mathbf{X}$}] (X) {};

   \draw[name path=image plane] (3cm,1.7cm) -- ++(0,-2*1.7cm);

   \draw[|-|] (0,-2cm) -- (3cm,-2cm) node[midway,below=1em] (f) {$f$};

   \draw[name path=ray] (C) -- (X);
   \fill[name intersections={of=ray and image plane}]
   (intersection-1) circle (2pt) coordinate (c);
   \node[label=above left:{$x$}] (x) at (c) {};

   \draw[dashed] (0,3cm) node[left=1em] {$Y$} -- (X) -- (7cm,0)
   node[below=1em]
   {$Z$};

   % \draw[|-|] (0,-3cm) -- (7cm,-3cm) node[midway,below=1em] (Z) {$Z$};

   \draw[|-|,name intersections={of=x-axis and image plane}] (x) ++(1em,0) --
   ($(intersection-1) + (1em,0)$) node [right=1em,midway] {$f \cdot
   \frac{Y}{Z}$};

\end{tikzpicture}

      \caption{Central projection for a pinhole camera}
   \label{pinhole}}
\end{figure}


When representing the points as homogeneous quantities, the central projection
can be expressed by a matrix multiplication. 
This can be written with homogeneous coordinates as
\begin{IEEEeqnarray}{rCl} \label{eq:projection}
      \left(
         \begin{array}{c}
            fX \\ fY \\ Z
         \end{array}
      \right) & = & \underbrace{\addstackgap[6pt]{\left(
         \begin{array}{cccc}
            f & 0 & 0 & 0 \\
            0 & f & 0 & 0 \\
            0 & 0 & 1 & 0
         \end{array}
\right)}}_{\text{Projection Matrix of $C$}} \left(\begin{array}{c} X \\ Y \\ Z \\ 1 \end{array}\right)
\end{IEEEeqnarray}

or in short
\begin{equation}
   \mbf{x} \sim P\mbf{X}
\end{equation}

% If the unit of the coordinatesystem is chosen to be $f$, then the image plane is
% $Z = f$ and 
% \eqref{eq:projection} becomes
% \begin{IEEEeqnarray}{rCl}
%       \left(
%          \begin{array}{c}
%             X \\ Y \\ \frac{z}{f}
%          \end{array}
%       \right) & = & \addstackgap[6pt]{\left(
%          \begin{array}{cccc}
%             1 & 0 & 0 & 0 \\
%             0 & 1 & 0 & 0 \\
%             0 & 0 & \frac{1}{f} & 0
%          \end{array}
% \right)} \left(\begin{array}{c} X \\ Y \\ Z \\ 1 \end{array}\right)
% \end{IEEEeqnarray}

\subsection{Camera Extrinsics}

The above situation is a special case wherein the camera centre $C$ defines the
origin and the optical and image axes are the coordinate axes. Thus, the rotation
and translation of the camera relative to this coordinate system is zero. More
generally, there might be a world coordinate frame with different origin 
and different axes, so that a coordinate transform must be applied to $\mbf{X}$
before the projection. 

Let $R \in \mathbb{R}^{3\times3}$ be a rotation matrix
giving the camera's rotation relative to the world frame and $t \in
\mathbb{R}^{3\times1}$ its translation such that

\begin{equation}
   \mbf{X}_{\text{cam}} = R \mbf{X}_{\text{world}} + t
\end{equation}

Then the projection of a point $\mbf{X}$
in world coordinates onto the image plane becomes

\begin{IEEEeqnarray}{rCl}
   \mbf{x} & = & P \mbf{X}\\
   \mbf{x} & = & \left(\begin{array}{ccc}
   f & 0 & 0 \\
   0 & f & 0 \\
   0 & 0 & 1
   \end{array}\right) \left[R \mid t\right] \mbf{X} \label{eq:projection_rt}
\end{IEEEeqnarray}

\subsection{Camera Intrinsics}
\label{subsec:intrinsics}

Above, the resulting image points $\mbf{x}$ were in \emph{normalised} image
coordinates. In particular, the
principal point---the
intersection of the image plane with the optical axis---was assumed to be
$(0,0)$. But generally, image coordinates are expressed in pixels relative to
the upper left corner of the sensor. To convert between normalised and pixel
coordinates, 
the camera's five intrinsic parameters and can be written in matrix form and
premultiplied in \eqref{eq:projection_rt} as
\begin{IEEEeqnarray}{rCl}
   \mbf{x} & = & \left(
   \begin{array}{ccc}
      s_x & s     & c_x \\
      0   & s_y   & c_y \\
      0   &       & 1
   \end{array} 
\right) \left(\begin{array}{ccc}
   f & 0 & 0 \\
   0 & f & 0 \\
   0 & 0 & 1
\end{array}\right) \left[R \mid t\right] \mbf{X}
\end{IEEEeqnarray}
where $s_x$ and $s_y$ are the focal lengths in $x$- and $y$-directions expressed
in pixel units per world unit (e.g. cm; $s_x$ and $s_y$ are not necessarily identical, if the sensor has
non-square pixels), $s$ the sensor skew (the pixels may not be rectangular;
their edges may not be perpendicular) which is usually zero, and the coordinates
of the principal point $(c_x,c_y)$ with respect to the origin of the image plane
which usually placed at the upper left corner.
The intrinsic camera parameters are assembled in
\begin{IEEEeqnarray}{rCl}
   K & = & \left(\begin{array}{ccc}
   fs_x & s & c_x \\
   0 & fs_y & c_y \\
   0 & 0 & 1
\end{array}\right)  
\end{IEEEeqnarray}
are therefore essential to
relate world points to image points which will be important for this
application. The normalised coordinates $\hat{\mbf{x}}$ for a pixel coordinate
$\mbf{x}$ can be computed as
\begin{equation}
   \hat{\mbf{x}} = K^{-1}\mbf{x},
\end{equation}
which will remove the effects of the calibration parameters and thus make the
image coordinates independent of the camera's internal characteristics.

In theory, these parameters could be obtained from the camera's
vendor who knows the precise manufacturing specifications. In practice, only the
focal lengths $f_x, f_y$ are known, in most cases only one with the assumption
of square pixels. Usually, the principal
point is assumed to be at the sensor centre and the pixels are assumed to be
rectangular. In practice however, there are variances introduced by different
causes such as imprecise manufacturing or physical impacts which may decentre
the lens such that the principal point is no longer at the centre. 

A further compilcation is introduced by the camera lens which will often have a
non-negligible distortion, most prominently radial distorion as depicted in
\autoref{distortion}. It can be modeled by the application of a 
distortion factor to the ideal undistorted image coordinates $(\tilde{x}, \tilde{y})$.

\begin{figure}
   {\centering      
      \begin{tikzpicture}[every to/.style={color=red}]

   \makegrayinprint

   \newcommand{\yshift}{6cm}
   \draw (0,0) rectangle (4,3);
   \begin{scope}[xshift=2cm,yshift=1.5cm]
      \draw (-1.5,-1) to[bend right=8] (1.5,-1) to[bend right=8] (1.5,1) to[bend
      right=8] (-1.5,1) to[bend right=8] (-1.5,-1);
   \end{scope}
   \begin{scope}[xshift=\yshift]
      \draw (0,0) rectangle (4,3);
      \begin{scope}[xshift=2cm,yshift=1.5cm]
         \draw (-1.5,-1) to (1.5,-1) to (1.5,1) to (-1.5,1) to (-1.5,-1);
      \end{scope}
   \end{scope}
\draw[-{Latex},thick,color=RoyalBlue,text=black] (4,1.5) ++(1mm,0) -- node[below,midway,font=\footnotesize]
   {correction} ($(\yshift,1.5) - (1mm,0)$);
\end{tikzpicture}

      \caption[Radial distortion]{Radially distorted image on the left, the corrected image on the
   right.}
   \label{distortion}}
\end{figure}


\begin{equation}
   \begin{pmatrix}
      x_d \\ y_d
   \end{pmatrix} = L(r)\begin{pmatrix}
      \tilde{x} \\ \tilde{y}
   \end{pmatrix}
\end{equation}

where $L$ is a nonlinear function of the distance $r$ from the distortion
center---usually coincident with the principal point. The function can be
approximated as an exponential with a Taylor expansion
\begin{equation}
   L(r) = 1 + \sum\limits_{i=1}^k \kappa_i r^i
\end{equation}
for a given $k$ \citep[see][ch. 7.4]{h&z2004}. The intrinsic camera
parameters which consist in the entries of $K$ and distortion coefficients
$\kappa_i$ must be determined in order to accurately relate world coordinates to
image coordinates. They can be found by calibrating the camera. Different
methods exist \citep[e.g][]{zhang2000} but will not be examined here.

\section{Epipolar Geometry}

\label{sec:epipolar}

Epipolar geometry is the geometry which relates the image points in two views of
the same scene. \autoref{epipolar} shows the basic setup. 

We consider a scene viewed by two cameras with optical centres $c_1$ and $c_2$,
where $c_1$ defines the origin,
world points $\mbf{X}_i\in\mathbb{R}^3$, where the subscript denotes the coordinate
frame---the first camera, arbitrarily chosen to be the left one, or the second
camera---and homogeneous image points $\mbf{x}_i$ which are the projections of $\mbf{X}_i$
onto the image planes and thus correspond to the same world point. The cameras
are related by a rigid body transform $(R,T)$, where $R$ is a $3\times3$
rotation matrix and $T$ the translation between the camera centres. Throughout
this work, the direction of coordinate frame transformation will be such that
\begin{equation}\label{eq:camera_transform}
   \mbf{X}_2 = R\mbf{X}_1 + T.
\end{equation}

It is obvious that the following relation holds,
\begin{equation}\label{eq:project_world_point}
   \lambda_i\mbf{x}_i = \mbf{X}_i, ~\lambda_i > 0
\end{equation}
that is, the world point lies on a ray through the optical centre and the image
point. 

\begin{figure}[h]
   {\centering      
      \usetikzlibrary{intersections, calc, arrows.meta}
\begin{tikzpicture}[every node/.style={outer sep=0,inner sep=0},
   label distance=5pt]
   \def\baseline{10cm}
   \draw[fill] (0,0) circle (2pt) node[label=below left:{$c_1$}] (c1) {};
   \draw[fill] (\baseline,0) circle (2pt) node[label=below left:{$c_2$}] (c2) {};

   \newcommand{\sensorplane}[1]{
      \path[name path=#1,draw] (1cm,0.5cm) -- (1cm,2.5cm) -- (3.5cm,1cm) -- (3.5cm,-1cm) -- cycle
   }
   \begin{scope}
      \sensorplane{plane1};
   \end{scope}

   \begin{scope}[yscale=1,xscale=-1,xshift=-\baseline]
      \sensorplane{plane2};
   \end{scope}

   \fill (0.5*\baseline,3cm) circle (2pt) node[label=above:{$\mathbf{X}_1$}]
   (X1) {};

   \draw (c1) -- ++(3cm,0) [fill] circle (2pt) node[label=above:{$e_1$}] (e1)
   {} coordinate (c);
   \draw[dotted] (c) -- (3.5cm,0) coordinate (c);
   \draw[solid] (c) -- (\baseline-3.5cm,0) coordinate (c);
   \draw[dotted] (c) -- (\baseline-3cm,0) coordinate (c);
   \draw[fill] (c) circle (2pt) node[label=above:{$e_2$}] (e2) {} -- (c2);

   % lines from camera center to x
   \path[name path=c1 to X1] (c1) -- (X1) coordinate[pos=.3] (c);
   \draw[solid,fill] (c1) -- (c) coordinate (x1) circle (2pt);
   \node[label=above:{$x_1$}] (x1) at (c) {};
   \draw[dotted,name intersections={of=c1 to X1 and plane1}] (c) --
   (intersection-2) coordinate (c);
   \draw[solid] (c) -- (X1);

   \path[name path=c2 to X1] (c2) -- (X1) coordinate[pos=.3] (c);
   \draw[solid,fill] (c2) -- (c) coordinate (x2) circle (2pt);
   \node[label=above:{$x_2$}] (x2) at (c) {};
   \draw[dotted,name intersections={of=c2 to X1 and plane2}] (c) --
   (intersection-2) coordinate (c);
   \draw[solid] (c) -- (X1);

   % draw epilines
   \draw[RoyalBlue,text=black] (e2) -- (x2) node[midway,label=above:{$l_2$}] (label-l2) {};

   \draw[RoyalBlue,text=black] (e1) -- (x1) node[midway,label=above:{$l_1$}] (label-l1) {};

   % draw relative rotation and translation
   \draw[-{Latex}] ($(c1) + (2em,-2em)$) to[bend right=50] node[below=1em, midway]
   {$R$, $T$} ($(c2) + (-2em,-2em)$);

\end{tikzpicture}

      \caption[Basic epipolar geometry]{Basic epipolar geometry with camera centres $c_1$, $c_2$, image
      points $\mbf{x}_1$, $\mbf{x}_2$, a world point $\mbf{X}_1$, epipoles
   $e_1$, $e_2$ and epipolar lines $l_1$, $l_2$}
   \label{epipolar}}
\end{figure}

Given the corresponding points $\mbf{x}_i$ in two images, the ultimate goal is
to retrieve the euclidean transform $(R,T)$.

In case the image coordinates for both cameras are normalised (c.f.
\autoref{subsec:intrinsics}), they have equal units, so
starting from \eqref{eq:camera_transform}, one can derive
\begin{IEEEeqnarray*}{rClls}
   \mbf{X}_2 & = & R\mbf{X}_1 + T & \hspace {2cm} \\
   \lambda_2\mbf{x}_2 & = & R\lambda_1\mbf{x}_1 + T & & (by \eqref{eq:project_world_point}) \\
   \lambda_2\widehat{T}\mbf{x}_2 & = & \widehat{T}R\lambda_1\mbf{x}_1 + \widehat{T}T & &
   $\widehat{T}\in\mathbb{R}^{3\times3}$ with $\widehat{T}\mbf{x} = T \times
   \mbf{x}$ \\
   \lambda_2\mbf{x}_2^T\widehat{T}\mbf{x}_2 & = &
   \mbf{x}_2^T\widehat{T}R\lambda_1\mbf{x}_1 + 0 & & $T \times T = 0$ \\
   \lambda_2\cdot 0 & = &
   \mbf{x}_2^T\widehat{T}R\lambda_1\mbf{x}_1 & & $\widehat{T}\mbf{x}_2$ is
   perpendicular to $\mbf{x}_2$ \\
   0 & = & \mbf{x}_2^T\underbrace{\widehat{T}R}_{E}\mbf{x}_1
   \label{eq:epiconstraint}\IEEEyesnumber
\end{IEEEeqnarray*}

The product $E=\widehat{T}R$ is the \emph{essential matrix} and the constraint it
imposes on corresponding image points the \emph{essential constraint}
\citep[see][ch. 5]{ma2003}. 

An intuition for the essential matrix can be
obtained from \autoref{epipolar}. Given an image point in one frame $\mbf{x}_1$,
in attempting to find the point $\mbf{x}_2$ corresponding to the same world
point $\mbf{X}_1$, the epipolar geometry restricts the search space to one
dimension---the epipolar line of $\mbf{x}_1$ in the second camera's image plane.
The camera centres and the world point define an epipolar plane. The
backprojection of $\mbf{x}_1$ is the ray through $\mbf{x}_1$ and the optical
centre $c_1$. All points on this ray are mapped to the same point on the image
plane of $c_2$. Depending on how far away the world point $\mbf{X}_1$ is from $c_1$, its
image on the second camera's image plane will vary---but it will be on the
intersection of the image plane and the epipolar plane, the epipolar line of
$\mbf{x}_1$. The line $l_2$ may be identified with its coimage (the orthogonal
complement of its preimage) $l_2 = e_2 \times \mbf{x}_2 \in \mathbb{R}^3$ so that

\begin{equation} \label{eq:epiline}
   \forall\mbf{x}: \mbf{x}\in l_2 \Leftrightarrow \mbf{x}\cdot l_2 = 0.
\end{equation} 

The coimage of the epipolar line is the vector perpendicular to the epipolar plane,
so every vector in this plane will have an inner product of $0$ with this
vector. Constrained to vectors in the image planes, this means that all vectors on the
epipolar line will have an inner product of $0$ with this vector $l_2$.
Referring back to \eqref{eq:epiconstraint}, it can be seen that multiplication
with $E$ will yield a term $\mbf{x}_2^TE = l_2$ which fulfils 
\begin{equation}
   \mbf{x}_1 \cdot l_2 = 0,
\end{equation}
which is precisely the relation stated in \eqref{eq:epiline}. The essential
matrix thus maps an image point onto its epipolar line in the other image.

\section{Essential Matrix Estimation}

Estimating the essential matrix between two
cameras and decomposing it into relative rotation and translation is a necessity
in the endeavour to communicate a necessary camera movement to the application's
user. The most prominent algorithms in this regard are the 8-Point algorithm
introduced in its original form by \citet{longuet-higgins1987} and improved upon
by \citet{hartley1997}, and the 5-Point algorithm proposed by
\citet{nister2004}. To illustrate the mathematical tractability of the problem,
the former will be presented below.

\subsection{The 8-Point Algorithm}\label{sec:eight-point}

The essential matrix has nine elements, but since all image coordinates are
projective quantities, the set of all essential matrices differing by a constant
factor is equivalent so that one degree of freedom is removed and at most eight
remain. If one were to formulate constraints on the elements as a system of
linear equations, eight of those should be sufficient to uniquely determine an
essential matrix. An improvement suggested in \citep{hartley1997} is the
preprocessing of the input data (the image coordinates) by translation and
scaling. This improves the robustness of the algorithm, but will be omitted
here.

For each point correspondence $\{\mbf{x}_1 = (x_1^i,y_1^i,1),\mbf{x}_2 = (x_2^i, y_2^i,
1)\}$, one linear equation
\begin{equation}
   \mbf{x}_2^T E \mbf{x}_1 = 0
\end{equation}
is generated, which can be rewritten as
\begin{IEEEeqnarray*}{rCcCcCc}
   0 & = & x_2^i x_1^i e_{11} & + & x_2^i y_1^i e_{12}  & + &  x_2^i e_{13} \\
     & + & y_2^i x_1^i e_{21} & + & y_2^i y_1^i e_{22} & + & y_2^i e_{23} \\
     & + & x_1^i e_{31} & + & y_1^i e_{32} & + & e_{33} \label{eq:linear_equation}
   \IEEEyesnumber
\end{IEEEeqnarray*}
Let $\mbf{e}$ denote the vector of $E$'s entries in row-major order, then
\begin{equation}
   0 = \left( x_2^i x_1^i, x_2^i y_1^i, x_2^i, y_2^i x_1^i, y_2^i y_1^i, y_2^i, x_1^i, y_1^i,
   1\right)\cdot\mbf{e}
\end{equation}

If $n$ correspondences are given, they each contribute one row to
\begin{IEEEeqnarray}{rCl}
   A\mbf{e} & = & \left[ \begin{array}{ccccccccc}
      x_2^1 x_1^1 & x_2^1 y_1^1 & x_2^1 & y_2^1 x_1^1 & y_2^1 y_1^1 & y_2^1 & x_1^1 & y_1^1 & 1 \\
   \vdots & & & & & & & & \vdots \\
      x_2^n x_1^n & x_2^n y_1^n & x_2^n & y_2^n x_1^n & y_2^n y_1^n & y_2^n & x_1^n & y_1^n & 1
\end{array} \right] \mbf{e} = 0 \nonumber\\*\label{eq:stacked_essential}
\end{IEEEeqnarray}

\newcommand{\norm}[1]{\left\lVert#1\right\rVert}

For eight noise-free point correspondences in non-degenerate general position, there is a
unique solution (up to scale) besides the trivial zero, but in practice, one
uses more correspondences and the system is overdetermined, so a
least-squares-solution minimising $\norm{ A\mbf{e} } = \sum_{ij}
(A\mbf{e})_{ij}^2$ is sought.  The solution is unique up to scale, since all
multiples of $\mbf{e}$ will satisfy \eqref{eq:stacked_essential}---this is also
the reason why the scale of the translation cannot be determined. One therefore
introduces the constraint $\norm{ \mbf{e} } = 1$ which also excludes a trivial
zero solution.

This solution vector is the singular
vector with the smallest singular value in the singular value decomposition of
$A$ or equivalently, the smallest eigenvector of $A^TA$
\citep[see][]{hartley1997}.

\subsection{Further Algorithms}

The 8-Point Algorithm is mathematically straightforward and linear, but in
practice it suffers from noise \citep[see][]{luong1993} and in real
applications approaches the robustness of other methods only in its normalised
form from \citep{hartley1997} (not related to \emph{normalised image
coordinates}).

It has been noted above that $E$ can have at most eight degrees of freedom. In
reality, it has only five, as rotation and translation have three degrees of
freedom each, and one is lost due to the indeterminable scale. In theory five
constraints from five pairs of image points thus are sufficient for finding $E$.
A solution was put forth by \citet{nister2004}, improved in
\citep{stewenius-engels-nister2006}. The algorithm is nonlinear and
thus much less easily understood and implemented, involving computing the roots
of a ten degree polynomial, and requires only five points, but can also be
applied to more. It can be considered a state-of-the-art solution to the
relative pose estimation problem; its performance in overdetermined cases in the
presence of noise compares favourably to other
direct methods requiring six \citep{pizarro2003}, seven or eight points.

Many other methods exist. Some---like the algorithms described above---find a
globally optimal solution in closed form---, while others employ heuristic
methods to iteratively approach a local optimum. A review is given in \citep{zhang1998}.

Direct methods such as the five-point or eight-point algorithms are frequently
used in schemes like RANSAC, which make the estimation more robust to outliers
(correspondences which are imprecise or incorrect). For a number of iterations,
a hypothesis for $E$ is computed on a minimal number of correspondences and then
evauluated on the whole data set. If the inliers in the data far outweigh the
outliers, it is probable that a noise-free subsample is selected. The best hypothesis is kept.
The simplified algorithm is shown in \autoref{alg:ransac}
\citep[c.f.][ch. 4.8]{h&z2004} and requires an error measure for $E$.


   \SetAlCapSkip{1ex}
\begin{algorithm}
   \caption{Simplified RANSAC scheme for essential matrix estimation}
   \label{alg:ransac}
   \KwData{$n$ point correspondences}
   \KwResult{a best-fitting essential matrix}
   Let $c_{\text{best}} \coloneqq 0$\;
   \For{$i \coloneqq 0$, $i < maxIter$}{
      Select randomly a minimal number of points to estimate $E_i$\;
      Compute error measure for $E_i$ on all $n$ points\;
      Let $C_i$ be the set of point pairs whose error does not exceed $\varepsilon$\;
      \If{$|C_i| > c_{\text{best}}$} {
         $c_{\text{best}} \coloneqq |C_i|$\;
         $E_{\text{best}} \coloneqq E_i$\;
      }
   }
   return $E_{\text{best}}$
\end{algorithm}

\section{Decomposing The Essential Matrix}

One step remains to recover the relative camera pose from corresponding points.
As per the derivation in \autoref{sec:epipolar}, the rotation and translation
between the two cameras is encoded in $E$. Given an essential matrix,
\citet[ch. 9.6]{h&z2004} show that there are four mathematically valid
decompositions of $E$ into $R$ and $T$, corresponding to four distinct
geometrical scenarios \citep[see][ch. 9.6]{h&z2004}. Only one of the
solutions will place a point $\mbf{X}_2 = R\mbf{X}_1 + T$ in front of both
cameras, the others cannot be realised in practice.  Triangulating one point
from its corresponding image points in the two views will therefore reveal the
one correct solution, with the translation scale unkown.
