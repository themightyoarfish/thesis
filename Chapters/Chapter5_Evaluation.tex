\chapter{Evaluation}
\label{ch:evaluation}

The approach has been evaluated on two realistic datasets which can be found in
appendix B. Computation was
performed off-line on a computer with the same C++ code running on the mobile device.
The most important questions are whether the direction of the necessary
translation is correctly identified and its scale decreasing with distance to
the target.  For both sets of images, the ground-truth translation between each
image and the first frame has been measured with centimetre accuracy, while the
ground-truth rotation has been estimated from manually labelled correspondence
as it is difficult to measure without the proper instruments. Since for the case
of noise-free correspondences in a non-degenerate configuration, relative pose
estimation algorithms are mathematically correct, this has been deemed
sufficiently accurate to evaluate the procedure. For each image pair, 19--27
correspondences have been labelled, of which the majority is used for pose
recovery. For pose recovery, RANSAC is used in conjunction with the five-point
solver, a point is considered an inlier for a given essential matrix if its
distance to its epipolar line is no more than three pixels, the confidence
threshold is $0.999$. These parameters
lead to the majority of points being inliers of the pose recovery, the few
outliers can be explained by imprecise labelling.

In both data sets, the translation was mostly in the horizontal direction and
along the optical axis; the vertical translation is thus neglected. Similarly,
rotation was applied mainly around the vertical axis.

In order to idealise the condition, the reference photograph has been used to
fill the role of the second frame for world scale computation. In reality, since
the reference location is unknown, the reference world scale is obtained from a
position somewhat off, thereby decreasing the accuracy of scale estimation.

$s$ refers to the scale at which the relative pose is computed in all plots.
Besides the full resolution, the images are scaled down by factor of $2$ (both
dimensions are halved, resulting in quarter size) and $4$ (sixteenth size). The
resolutions evaluated are thus $3264\times2448$, $1632\times1224$, and
$816\times612$.

The following graphics illustrate three things
\begin{enumerate}
   \item The difference between the computed necessary rotation and the actually
      necessary one
   \item The difference in direction of the computed necessary translation and
      the actually necessary one
   \item The correlation between the true ratio of distances obtained by
      measuring camera movement and the average distance ratio computed with first
      and second (or in this case reference) frames based on automatic feature matching, at
      three different scales
\end{enumerate}

\section{Train Station Data Set}

In this series, the camera was moved horizontally from the reference to the
right while also coming closer to the building.
A schematic bird's eye view of the captures is shown in \autoref{fig:train_scene}.

\begin{figure}[!htbp]
   {\centering      
      \begin{tikzpicture}
   \begin{axis}[
         xmin = -1,
         xmax = 6,
         ymin = -1,
         ymax = 6,
         xlabel = $x$,
         ylabel = $y$,
         every node near coord/.append style={yshift=-0.5cm,anchor=-10} % hacky way putting nodes below points instead of above
      ]
      \addplot+[
         nodes near coords,
         only marks,
         point meta=explicit symbolic,
      ]
      coordinates {
         (0,0)        [0]
         (5.65,2.65)  [1]
         (3.12,0.16)  [2]
         (0.83,0.03)  [3]
         (0.20,0.03)  [4]
         (2,5)        [station]
      };

      \coordinate (station) at (axis cs:2,5);

      \foreach \x / \y in {0/0,5.65/2.65,3.12/0.16,0.83/0.03,0.20/0.03}
      {
         \edef\temp{
            \noexpand\node[outer sep=0,inner sep=0] (foo) at (axis cs:\x,\y) {}; 
         \noexpand\draw[->] (foo) -- ($ (foo) !1cm! (station) $);
      }
         \temp
      }

   \end{axis}
\end{tikzpicture}

      \caption[Schematic of the train scene]{Schematic representation of the Train Station data set. Lengths and angles are not
      precise.}
   \label{fig:train_scene}}
\end{figure}
\autoref{tab:train_data} summarises the ground truth for the five images.

\subsection{Scale Estimation}

\newcolumntype{m}{>{$}c<{$}} % math column
\rowcolors{2}{gray!5}{white} % alternating colours

\begin{table}
   \caption[Train data ground truth]{Ground truth for the train data. Image 0 is the reference frame,
      translations and rotations are given as in \eqref{eq:camera_transform}
   relative to the reference frame.}
   \begin{tabular}{cmmm}
      \toprule
      \rowcolor{white}
      Image        & \text{Relative translation} & \text{Relative Rotation} & \text{ratio}\\
      number       & [x,y,z]                         & [\theta_x, \theta_y, \theta_z]
      \\
      \midrule
      0 & [0      , 0 , 0]      & [0        , 0       , 0]       & 1      \\
      1 & [ .9053 , 0 , .4246 ] & [ -3.3779 , -9.3779 , 1.05121] & 3.8936 \\
      2 & [ .9986 , 0 , .0512 ] & [ -1.3274 , -5.7134 , -.1884 ] & 1.6461 \\
      3 & [ .9993 , 0 , .0361 ] & [ -1.7156 , -2.4761 , .3469  ] & 1.0965 \\
      4 & [ .9950 , 0 , .0995 ] & [ .054606 , -4.4867 , .2452  ] & 1.0343 \\\bottomrule
   \end{tabular}
   \label{tab:train_data}
\end{table}

\autoref{fig:train_dist_ratio} shows how the average distance of points to the
first frame's camera varies with the second image used for triangulation. The
plot illustrates that---especially at full resolution---the ratio based on
feature matching closely mirrors the real value. The difference increases with
decreasing image scale, but the slope of the graphs is quite similar. This
shows that indeed with increasing distance to the first frame, the ratio
decreases, allowing a deduction as to how close the camera is to the target, at
least with respect to previous iterations, which is the primary objective. 
The decrease in ratio closely correlates with the decrease in distance which
is apparent on inspection of \autoref{fig:train_scene}. For instance, the viewpoints
3 and 4 are much closer together than e.g. 2 and 3, and the difference in ratios
is much smaller between 3 and 4 as well. 

The correlation is higher for AKAZE
features than for SIFT ones, where a strong spike for image 1 can be observed.
For SIFT, the decrease of ratio between images 2 and 3 is also hardly visible at
$s=2$. The unusual spike for image 1 poses the problem that the visualisation
would tell the user that they need to move disproportionally far compared to
the other images. Since the error in this case is corrected for the next image,
this may not be a big problem, but will affect user experience. For SIFT
features, one can also observe that the scale of the images appears to be less
relevant, possibly an indication for the better scale invariance of the
descriptor.

Generally, it can be concluded that on this data set, AKAZE features are an
appropriate means of estimating the scale of relative camera translation.

\begin{figure}
   \begin{subfigure}{.5\linewidth}
      \centering      
      \pgfplotstableread[col sep=space]{data/bahnhof_detector_KAZE_resize_1_ratio_0.800000.dat}\datatablenoresize
\pgfplotstableread[col sep=space]{data/bahnhof_detector_KAZE_resize_2_ratio_0.800000.dat}\datatableresize
\pgfplotstableread[col sep=space]{data/bahnhof_detector_KAZE_resize_4_ratio_0.800000.dat}\datatabledoubleresize
\begin{tikzpicture}
   \begin{axis}[
         xlabel={Image Number},
         ylabel={Distance ratio},
         ymax=4.5,
         xtick=data,
         % xticklabels from table={\datatable}{fname}
         scale=.6, % dare you to change this to sth larger
         legend pos=north east,
      ]

      % original size
      \addplot table[
         x expr=\coordindex,
         y=realdist_ratio,
      ] {\datatablenoresize};
      \addlegendentry{Truth}

      \addplot table[
         x expr=\coordindex,
         y=dist_ratio,
      ] {\datatablenoresize};
      \addlegendentry{$s=1$}

      % scaled by 2
      \addplot table[
         x expr=\coordindex,
         y=dist_ratio,
      ] {\datatableresize};
      \addlegendentry{$s=2$}

      % scaled by 2
      \addplot table[
         x expr=\coordindex,
         y=dist_ratio,
      ] {\datatabledoubleresize};
      \addlegendentry{$s=4$}

   \end{axis}
\end{tikzpicture}

      \caption{AKAZE features}
      \label{fig:train_KAZE_dist_ratio}
   \end{subfigure}
   \quad
   \begin{subfigure}{.5\linewidth}
      \centering      
      \pgfplotstableread[col sep=space]{data/bahnhof_detector_SIFT_resize_1_ratio_0.800000.dat}\datatablenoresize
\pgfplotstableread[col sep=space]{data/bahnhof_detector_SIFT_resize_2_ratio_0.800000.dat}\datatableresize
\pgfplotstableread[col sep=space]{data/bahnhof_detector_SIFT_resize_4_ratio_0.800000.dat}\datatabledoubleresize
\begin{tikzpicture}
   \begin{axis}[
         xlabel=\vphantom{Image Number},
         % ylabel={Distance ratio},
         % ymax=11, % changing this to 10 will fuck up the subfigure. dafug.
         yticklabel pos=right,
         xtick=data,
         ymode=log,
         % xticklabels from table={\datatable}{fname},
         legend pos=north east,
         scale=.6,
      ]

      % original size
      \addplot table[
         x expr=\coordindex,
         y=realdist_ratio,
      ] {\datatablenoresize};
      \addlegendentry{Ground truth}

      \addplot+[dashed,mark=o] table[
         x expr=\coordindex,
         y=dist_ratio,
      ] {\datatablenoresize};
      \addlegendentry{$s=1$}

      % scaled by 2
      \addplot+[dashed,mark=o] table[
         x expr=\coordindex,
         y=dist_ratio,
      ] {\datatableresize};
      \addlegendentry{$s=2$}

      % scaled by 4
      \addplot+[dashed,mark=o] table[
         x expr=\coordindex,
         y=dist_ratio,
      ] {\datatabledoubleresize};
      \addlegendentry{$s=4$}

   \end{axis}
\end{tikzpicture}




      \caption{SIFT features}
      \label{fig:train_SIFT_dist_ratio}
   \end{subfigure}
   \caption[Train data: Distance ratio]{Train Station data set: Evolution of the distance ratio between images}
   \label{fig:train_dist_ratio}
\end{figure}

\subsection{Rotation Estimation}

\autoref{fig:train_rotation_KAZE} and \autoref{fig:train_rotation_SIFT} illustrate the
difference between the actually necessary camera rotation and the computed one
for AKAZE and SIFT features, respectively. Rotations about the optical an X axes
are small and thus not very interesting and the deviation is small. 

Focusing on the Y-rotation, it is obvious is that the estimation
quality decreases especially for $s=4$, but the difference does not exceed 5
degrees and thus the estimate is very usable, especially since for reasonably
quick updates, mostly the direction of necessary rotation is important, not the
absolute magnitude.

The performance of SIFT is even better for scales $s=1$ and $s=2$, but slightly
worse on the smallest scale (see \autoref{fig:train_SIFT_rotation_4}).

\begin{figure}
   \begin{subfigure}{.5\linewidth}
      \centering      
      \pgfplotstableread[col sep=space]{data/bahnhof_detector_KAZE_resize_1_ratio_0.800000.dat}\datatable
\begin{tikzpicture}
   \begin{axis}[
         xlabel={Image Number},
         ylabel={Rotation angles},
         xtick=data,
         cycle list name=mylist,
         scale=.7,
      ]

      % original size
      \addplot table[
         x expr=\coordindex,
         y=realthetax,
      ] {\datatable};
      \addlegendentry{True $\theta_x$}
      \addplot table[
         x expr=\coordindex,
         y=realthetay,
      ] {\datatable};
      \addlegendentry{True $\theta_y$}
      \addplot table[
         x expr=\coordindex,
         y=realthetaz,
      ] {\datatable};
      \addlegendentry{True $\theta_z$}

      % half size
      \addplot table[
         x expr=\coordindex,
         y=thetax,
      ] {\datatable};
      \addlegendentry{$\theta_x$}
      \addplot table[
         x expr=\coordindex,
         y=thetay,
      ] {\datatable};
      \addlegendentry{$\theta_y$}
      \addplot table[
         x expr=\coordindex,
         y=thetaz,
      ] {\datatable};
      \addlegendentry{$\theta_z$}

   \end{axis}
\end{tikzpicture}



      \caption{$s=1$}
      \label{fig:train_KAZE_rotation_1}
   \end{subfigure}
   \quad
   \begin{subfigure}{.5\linewidth}
      \centering      
      \pgfplotstableread[col sep=space]{data/bahnhof_detector_KAZE_resize_2_ratio_0.800000.dat}\datatable
\begin{tikzpicture}
   \begin{axis}[
         xlabel={\vphantom{Image Number}},
         xtick=data,
         cycle list name=mylist,
         scale=.7,
         yticklabels={},
      ]

      % original size
      \addplot table[
         x expr=\coordindex,
         y=realthetax,
      ] {\datatable};
      \addlegendentry{True $\theta_x$}
      \addplot table[
         x expr=\coordindex,
         y=realthetay,
      ] {\datatable};
      \addlegendentry{True $\theta_y$}
      \addplot table[
         x expr=\coordindex,
         y=realthetaz,
      ] {\datatable};
      \addlegendentry{True $\theta_z$}

      % half size
      \addplot table[
         x expr=\coordindex,
         y=thetax,
      ] {\datatable};
      \addlegendentry{$\theta_x$}
      \addplot table[
         x expr=\coordindex,
         y=thetay,
      ] {\datatable};
      \addlegendentry{$\theta_y$}
      \addplot table[
         x expr=\coordindex,
         y=thetaz,
      ] {\datatable};
      \addlegendentry{$\theta_z$}

   \end{axis}
\end{tikzpicture}




      \caption{$s=2$}
      \label{fig:train_KAZE_rotation_2}
   \end{subfigure}\\[3ex]
   \begin{subfigure}{\linewidth}
      \centering      
      \pgfplotstableread[col sep=space]{data/bahnhof_detector_KAZE_resize_4_ratio_0.800000.dat}\datatable
\begin{tikzpicture}
   \begin{axis}[
         scale=.7,
         cycle list name=mylist,
         xtick=data,
      ]

      % original size
      \addplot table[
         x expr=\coordindex,
         y=realthetax,
      ] {\datatable};
      \addlegendentry{True $\theta_x$}
      \addplot table[
         x expr=\coordindex,
         y=realthetay,
      ] {\datatable};
      \addlegendentry{True $\theta_y$}
      \addplot table[
         x expr=\coordindex,
         y=realthetaz,
      ] {\datatable};
      \addlegendentry{True $\theta_z$}

      % half size
      \addplot table[
         x expr=\coordindex,
         y=thetax,
      ] {\datatable};
      \addlegendentry{$\theta_x$}
      \addplot table[
         x expr=\coordindex,
         y=thetay,
      ] {\datatable};
      \addlegendentry{$\theta_y$}
      \addplot table[
         x expr=\coordindex,
         y=thetaz,
      ] {\datatable};
      \addlegendentry{$\theta_z$}

   \end{axis}
\end{tikzpicture}





      \caption{$s=4$}
      \label{fig:train_KAZE_rotation_4}
   \end{subfigure}
   \caption[Train data: Rotation AKAZE]{Train Station data set: Angles of rotation relative to reference with
   AKAZE features on full, quarter and sixteenth resolution}
   \label{fig:train_rotation_KAZE}
\end{figure}

\begin{figure}
   \begin{subfigure}{.5\linewidth}
      \centering      
      \pgfplotstableread[col sep=space]{data/bahnhof_detector_SIFT_resize_1_ratio_0.800000.dat}\datatable
\begin{tikzpicture}
   \begin{axis}[
         xlabel={Image Number},
         ylabel={Rotation angles},
         xtick=data,
         cycle list name=mylist,
         scale=.7,
      ]

      % original size
      \addplot table[
         x expr=\coordindex,
         y=realthetax,
      ] {\datatable};
      \addlegendentry{True $\theta_x$}
      \addplot table[
         x expr=\coordindex,
         y=realthetay,
      ] {\datatable};
      \addlegendentry{True $\theta_y$}
      \addplot table[
         x expr=\coordindex,
         y=realthetaz,
      ] {\datatable};
      \addlegendentry{True $\theta_z$}

      % half size
      \addplot table[
         x expr=\coordindex,
         y=thetax,
      ] {\datatable};
      \addlegendentry{$\theta_x$}
      \addplot table[
         x expr=\coordindex,
         y=thetay,
      ] {\datatable};
      \addlegendentry{$\theta_y$}
      \addplot table[
         x expr=\coordindex,
         y=thetaz,
      ] {\datatable};
      \addlegendentry{$\theta_z$}

   \end{axis}
\end{tikzpicture}



      \caption{$s=1$}
      \label{fig:train_SIFT_rotation_1}
   \end{subfigure}
   \quad
   \begin{subfigure}{.5\linewidth}
      \centering      
      \pgfplotstableread[col sep=space]{data/bahnhof_detector_SIFT_resize_2_ratio_0.800000.dat}\datatable
\begin{tikzpicture}
   \begin{axis}[
         xlabel={\vphantom{Image Number}},
         xtick=data,
         cycle list name=mylist,
         scale=.7,
         yticklabels={},
         legend pos=south east,
         ymin=-10,
         ymax=1,
      ]

      % original size
      \addplot table[
         x expr=\coordindex,
         y=realthetax,
      ] {\datatable};
      \addlegendentry{True $\theta_x$}
      \addplot table[
         x expr=\coordindex,
         y=realthetay,
      ] {\datatable};
      \addlegendentry{True $\theta_y$}
      \addplot table[
         x expr=\coordindex,
         y=realthetaz,
      ] {\datatable};
      \addlegendentry{True $\theta_z$}

      % half size
      \addplot table[
         x expr=\coordindex,
         y=thetax,
      ] {\datatable};
      \addlegendentry{$\theta_x$}
      \addplot table[
         x expr=\coordindex,
         y=thetay,
      ] {\datatable};
      \addlegendentry{$\theta_y$}
      \addplot table[
         x expr=\coordindex,
         y=thetaz,
      ] {\datatable};
      \addlegendentry{$\theta_z$}

   \end{axis}
\end{tikzpicture}




      \caption{$s=2$}
      \label{fig:train_SIFT_rotation_2}
   \end{subfigure}\\[3ex]
   \begin{subfigure}{\linewidth}
      \centering      
      \pgfplotstableread[col sep=space]{data/bahnhof_detector_SIFT_resize_4_ratio_0.800000.dat}\datatable
\begin{tikzpicture}
   \begin{axis}[
         scale=.7,
         cycle list name=mylist,
         xtick=data,
      ]

      % original size
      \addplot table[
         x expr=\coordindex,
         y=realthetax,
      ] {\datatable};
      \addlegendentry{True $\theta_x$}
      \addplot table[
         x expr=\coordindex,
         y=realthetay,
      ] {\datatable};
      \addlegendentry{True $\theta_y$}
      \addplot table[
         x expr=\coordindex,
         y=realthetaz,
      ] {\datatable};
      \addlegendentry{True $\theta_z$}

      % half size
      \addplot table[
         x expr=\coordindex,
         y=thetax,
      ] {\datatable};
      \addlegendentry{$\theta_x$}
      \addplot table[
         x expr=\coordindex,
         y=thetay,
      ] {\datatable};
      \addlegendentry{$\theta_y$}
      \addplot table[
         x expr=\coordindex,
         y=thetaz,
      ] {\datatable};
      \addlegendentry{$\theta_z$}

   \end{axis}
\end{tikzpicture}





      \caption{$s=4$}
      \label{fig:train_SIFT_rotation_4}
   \end{subfigure}
   \caption[Train data: Rotation SIFT]{Train Station data set: Angles of rotation relative to reference with SIFT features}
   \label{fig:train_rotation_SIFT}
\end{figure}

\subsection{Translation Estimation}

Finally and most importantly, the directions of the necessary translation must
be evaluated. \autoref{fig:train_direction} plots the angular difference in
degree between the actual necessary translation and the computed one. The
reference frame $0$ is omitted since the translation vector to compare with is
$(0,0,0)$.

\begin{figure}
   \begin{subfigure}[b]{.5\linewidth}
      \centering      
      \pgfplotstableread[col sep=space]{data/bahnhof_detector_KAZE_resize_1_ratio_0.800000.dat}\datatablenoresize
\pgfplotstableread[col sep=space]{data/bahnhof_detector_KAZE_resize_2_ratio_0.800000.dat}\datatableresize
\pgfplotstableread[col
sep=space]{data/bahnhof_detector_KAZE_resize_4_ratio_0.800000.dat}\datatabledoubleresize
\begin{tikzpicture}
   \begin{axis}[
         xlabel={Image Number},
         ylabel={Angular difference},
         xtick=data,
         legend pos=north east,
         scale=.6,
      ]
      \pgfplotstablegetrowsof{\datatablenoresize}
      \pgfmathsetmacro\yfin{\pgfmathresult - 4}
      \pgfmathsetmacro\yini{0}


      % original size
      \addplot table[
         skip coords between index={\yini}{\yfin},
         x expr=\coordindex,
         y expr={
            acos(
            (
            \thisrow{x}*\thisrow{realx} +
            \thisrow{y}*\thisrow{realy} +
            \thisrow{z}*\thisrow{realz}
            ) / 
            (
            sqrt(
            (\thisrow{x})^2 +
            (\thisrow{y})^2 +
            (\thisrow{z})^2
            ) * 
            sqrt(
            (\thisrow{realx})^2 +
            (\thisrow{realy})^2 +
            (\thisrow{realz})^2
            )
            )
            )
         }
      ] {\datatablenoresize};
      \addlegendentry{$s=1$}

      % quarter size
      \addplot table[
         skip coords between index={\yini}{\yfin},
         x expr=\coordindex,
         y expr={
            acos(
            (
            \thisrow{x}*\thisrow{realx} +
            \thisrow{y}*\thisrow{realy} +
            \thisrow{z}*\thisrow{realz}
            ) / 
            (
            sqrt(
            (\thisrow{x})^2 +
            (\thisrow{y})^2 +
            (\thisrow{z})^2
            ) * 
            sqrt(
            (\thisrow{realx})^2 +
            (\thisrow{realy})^2 +
            (\thisrow{realz})^2
            )
            )
            )
         }
      ] {\datatableresize};
      \addlegendentry{$s=2$}

      % sixteenth size
      \addplot table[
         skip coords between index={\yini}{\yfin},
         x expr=\coordindex,
         y expr={
            acos(
            (
            \thisrow{x}*\thisrow{realx} +
            \thisrow{y}*\thisrow{realy} +
            \thisrow{z}*\thisrow{realz}
            ) / 
            (
            sqrt(
            (\thisrow{x})^2 +
            (\thisrow{y})^2 +
            (\thisrow{z})^2
            ) * 
            sqrt(
            (\thisrow{realx})^2 +
            (\thisrow{realy})^2 +
            (\thisrow{realz})^2
            )
            )
            )
         }
      ] {\datatabledoubleresize};
      \addlegendentry{$s=4$}
   \end{axis}
\end{tikzpicture}

      \caption{AKAZE features}
      \label{fig:train_KAZE_direction}
   \end{subfigure}
   \begin{subfigure}[b]{.5\linewidth}
      \centering      
      \pgfplotstableread[col sep=space]{data/bahnhof_detector_SIFT_resize_1_ratio_0.800000.dat}\datatablenoresize
\pgfplotstableread[col sep=space]{data/bahnhof_detector_SIFT_resize_2_ratio_0.800000.dat}\datatableresize
\pgfplotstableread[col
sep=space]{data/bahnhof_detector_SIFT_resize_4_ratio_0.800000.dat}\datatabledoubleresize
\begin{tikzpicture}
   \begin{axis}[
         xlabel=\vphantom{Image Number},
         % ylabel={Angular difference},
         yticklabels={},
         xtick=data,
         legend pos=north east,
         scale=.6,
         ymin=80,
         ymax=120,
      ]
      \pgfplotstablegetrowsof{\datatablenoresize}
      \pgfmathsetmacro\yfin{\pgfmathresult - 4}
      \pgfmathsetmacro\yini{0}


      % original size
      \addplot table[
         skip coords between index={\yini}{\yfin},
         x expr=\coordindex,
         y expr={
            acos(
            (
            \thisrow{x}*\thisrow{realx} +
            \thisrow{y}*\thisrow{realy} +
            \thisrow{z}*\thisrow{realz}
            ) / 
            (
            sqrt(
            (\thisrow{x})^2 +
            (\thisrow{y})^2 +
            (\thisrow{z})^2
            ) * 
            sqrt(
            (\thisrow{realx})^2 +
            (\thisrow{realy})^2 +
            (\thisrow{realz})^2
            )
            )
            )
         }
      ] {\datatablenoresize};
      \addlegendentry{$s=1$}

      % quarter size
      \addplot table[
         skip coords between index={\yini}{\yfin},
         x expr=\coordindex,
         y expr={
            acos(
            (
            \thisrow{x}*\thisrow{realx} +
            \thisrow{y}*\thisrow{realy} +
            \thisrow{z}*\thisrow{realz}
            ) / 
            (
            sqrt(
            (\thisrow{x})^2 +
            (\thisrow{y})^2 +
            (\thisrow{z})^2
            ) * 
            sqrt(
            (\thisrow{realx})^2 +
            (\thisrow{realy})^2 +
            (\thisrow{realz})^2
            )
            )
            )
         }
      ] {\datatableresize};
      \addlegendentry{$s=2$}

      % sixteenth size
      \addplot table[
         skip coords between index={\yini}{\yfin},
         x expr=\coordindex,
         y expr={
            acos(
            (
            \thisrow{x}*\thisrow{realx} +
            \thisrow{y}*\thisrow{realy} +
            \thisrow{z}*\thisrow{realz}
            ) / 
            (
            sqrt(
            (\thisrow{x})^2 +
            (\thisrow{y})^2 +
            (\thisrow{z})^2
            ) * 
            sqrt(
            (\thisrow{realx})^2 +
            (\thisrow{realy})^2 +
            (\thisrow{realz})^2
            )
            )
            )
         }
      ] {\datatabledoubleresize};
      \addlegendentry{$s=4$}
   \end{axis}
\end{tikzpicture}


      \caption{SIFT features}
      \label{fig:train_SIFT_direction}
   \end{subfigure}
   \caption[Train data: Translation]{Train Station data set: Angular difference between actually necessary translation and algorithmic estimate}
   \label{fig:train_direction}
\end{figure}

% It is obvious that the estimates are completely useless, the difference
% exceeds $80$ degrees in all cases. The results are so staggeringly bad as to
% be suggestive of conceptual or mathematical error, though none could be found.
% With these estimates, the user will be sent into an entirely wrong direction.
It is obvious that the estimates are completely useless, the difference exceeds
$80$ degrees in all cases. With these estimates, the user will be sent into an
entirely wrong direction.

\FloatBarrier

\section{Manor Data Set}

Seven images (including the reference photo) have been taken with movement to
the right and backwards as well as forwards. The motif was always centred in the
frame, thus there is prominent rotation around the $y$-Axis. 
The schematic positions for the seven manor captures (including reference
photograph) are shown in \autoref{fig:manor_scene}. In contrast to the
train station set, there is more significant movement along the optical axis.

The ground truth data is summarised in \autoref{tab:manor_data}.

\begin{figure}
   {\centering      
      \begin{tikzpicture}
   \begin{axis}[
         xmin = -2,
         xmax = 20,
         ymin = -10,
         ymax = 16,
         xlabel = $x$,
         ylabel = $y$,
         every node near coord/.append style={anchor=5} % hacky way putting nodes below points instead of above
      ]
      \addplot+[
         nodes near coords,
         only marks,
         point meta=explicit symbolic,
         mark=*,
         black,
      ]
      coordinates {
         (0,0)   [0]
         (2,0)   [1]
         (4,2)   [2]
         (4,4)   [3]
         (8,-5)  [4]
         (10,4)  [5]
         (16,-6) [6]
         (16,2)  [first frame]
         (10,15) [manor]
      };

      \coordinate (manor) at (axis cs:10,15);

      \foreach \x / \y in {0/0,2/0,4/2,4/4,8/-5,10/4,16/-6,16/2}
      {
         \edef\temp{
            \noexpand\node[outer sep=0,inner sep=0] (foo) at (axis cs:\x,\y) {}; 
            \noexpand\draw[->] (foo) -- ($ (foo) !.5cm! (manor) $);
         }
         \temp
      }

   \end{axis}
\end{tikzpicture}

      \caption[Schematic of the manor data set]{Schematic representation of the Manor data set. Lengths and angles are not
      precise.}
   \label{fig:manor_scene}}
\end{figure}


\begin{table}
   \caption[Manor data ground truth]{Ground truth for the manor data. Image $0$ is the reference frame,
      translations and rotations are given as in \eqref{eq:camera_transform}
   relative to the reference frame.}
   \begin{tabular}{cmmm}
      \toprule
      \rowcolor{white}
      Image        & \text{Relative translation} & \text{Relative Rotation} & \text{ratio}\\
      number       & [x,y,z]                         & [\theta_x, \theta_y, \theta_z]
      \\
      \midrule
      0 & [0       , 0  , 0]        & [0 , 0        , 0]                & 1       \\
      1 & [ 1      , 0  , 0       ] & [ -1.7857   , -5.4827  , 2.1073 ] & 1.1401  \\
      2 & [ 0.8944 , 0. , 0.4472  ] & [ -2.1428   , -6.5773  , 1.6584 ] & 1.3437  \\
      3 & [ 0.7071 , 0. , 0.7071  ] & [ 0.7263    , -5.0686  , 2.6176 ] & 1.3254  \\
      4 & [ 0.8479 , 0. , -0.5299 ] & [ -1.4146   , -10.7998 , 2.2250 ] & 1.5168  \\
      5 & [ 0.9284 , 0. , 0.3713  ] & [ -0.1887   , -16.6670 , 1.2211 ] & 2.5495  \\
      6 & [ 0.9363 , 0. , -0.3511 ] & [ -0.8725   , -18.0933 , 1.5385 ] & 2.0155  \\\bottomrule
   \end{tabular}
   \label{tab:manor_data}
\end{table}

\subsection{Scale Estimation}

The evolution of the translation scale is shown in \autoref{fig:manor_dist_ratio}.
It is apparent that the movement purely along the optical axis between images
$2$ and $3$ is a problem. As the real distance to the target marginally
increases, so should the ratio, but it decreases instead. Frames $5$ and $6$
illustrate a problem with the scale estimation procedure itself. For it to work
precisely, only movement along the line between first and reference frames is
assumed, as a decreased distance to the first frame is interpreted as an
increased distance to the reference frame, which is not necessarily the case as
shown here. Even the ``ground truth'' computed from actual camera distances is thus
of limited use.

For the AKAZE descriptor, only the full resolution comes reasonably close in
magnitude and somewhat in slope. Wit SIFT, the slope is more accurately
reproduced with $s=2$, but strangely less accurately on full resolution. It is
possible that the reduction in noise brought about by downsampling can improve
the estimate, but the other experiments do not show it. For the
smallest scale, the estimate degenerates strongly.

Generally is can be stated that the estimates are less close than those for the
train data set, but also that large movement along the optical axis shows the
limits of this simple approach at scale estimation. Realistically, the user will
not move as erratically so this kind of scenario is extreme.

\begin{figure}
   \begin{subfigure}[b]{.5\linewidth}
      \centering      
      \pgfplotstableread[col sep=space]{data/gut_rosenkrantz_detector_KAZE_resize_1_ratio_0.800000.dat}\datatablenoresize
\pgfplotstableread[col sep=space]{data/gut_rosenkrantz_detector_KAZE_resize_2_ratio_0.800000.dat}\datatableresize
\pgfplotstableread[col sep=space]{data/gut_rosenkrantz_detector_KAZE_resize_4_ratio_0.800000.dat}\datatabledoubleresize
\begin{tikzpicture}
   \begin{axis}[
         xlabel={Image Number},
         ylabel={Distance ratio},
         % xtick=data,
         % xticklabels from table={\datatable}{fname},
         legend pos=north west,
      ]

      % original size
      \addplot table[
         x expr=\coordindex,
         y=realdist_ratio,
      ] {\datatablenoresize};
      \addlegendentry{Ground truth}

      \addplot table[
         x expr=\coordindex,
         y=dist_ratio,
      ] {\datatablenoresize};
      \addlegendentry{$s=1$}

      % scaled by 2
      \addplot table[
         x expr=\coordindex,
         y=dist_ratio,
      ] {\datatableresize};
      \addlegendentry{$s=2$}

      % scaled by 4
      \addplot table[
         x expr=\coordindex,
         y=dist_ratio,
      ] {\datatabledoubleresize};
      \addlegendentry{$s=4$}

   \end{axis}
\end{tikzpicture}


      \caption{AKAZE features}
      \label{fig:manor_KAZE_dist_ratio}
   \end{subfigure}
   \quad\begin{subfigure}[b]{.5\linewidth}
      \centering      
      \pgfplotstableread[col sep=space]{data/gut_rosenkrantz_detector_SIFT_resize_1_ratio_0.800000.dat}\datatablenoresize
\pgfplotstableread[col sep=space]{data/gut_rosenkrantz_detector_SIFT_resize_2_ratio_0.800000.dat}\datatableresize
\pgfplotstableread[col sep=space]{data/gut_rosenkrantz_detector_SIFT_resize_4_ratio_0.800000.dat}\datatabledoubleresize
\begin{tikzpicture}
   \begin{axis}[
         xlabel=\vphantom{Image Number},
         yticklabels={},
         ymin=-0.5,
         ymax=7,
         xtick=data,
         legend pos=north west,
         scale=.7,
      ]

      % original size
      \addplot table[
         x expr=\coordindex,
         y=realdist_ratio,
      ] {\datatablenoresize};
      \addlegendentry{Truth}

      \addplot table[
         x expr=\coordindex,
         y=dist_ratio,
      ] {\datatablenoresize};
      \addlegendentry{$s=1$}

      % scaled by 2
      \addplot table[
         x expr=\coordindex,
         y=dist_ratio,
      ] {\datatableresize};
      \addlegendentry{$s=2$}

      % scaled by 4
      \addplot table[
         x expr=\coordindex,
         y=dist_ratio,
      ] {\datatabledoubleresize};
      \addlegendentry{$s=4$}

   \end{axis}
\end{tikzpicture}



      \caption{SIFT features}
      \label{fig:manor_SIFT_dist_ratio}
   \end{subfigure}
   \caption[Manor Data: distance ratio]{Manor data set: Evolution of the distance ratio between images}
   \label{fig:manor_dist_ratio}
\end{figure}

\subsection{Rotation Estimation}

\autoref{fig:manor_KAZE_rotation} and \autoref{fig:manor_SIFT_rotation}
illustrate how accurately the necessary rotation is computed. On this data, AKAZE
outperforms SIFT with default parameters (see \autoref{tab:akaze_params}). On both full and
half scale, there is negligible deviation from the truth, but on quarter scale,
there are more than $5$ degrees of difference and a complete failure for frame
$3$ (the direction is wrong, not only the magnitude).

\rowcolors{2}{gray!5}{white} % alternating colours
\begin{table}
   \begin{center}
      \begin{tabular}{>{\ttfamily}ll}
         \rowcolor{white}
         \toprule
         \rmfamily Parameter     & Value \\
         \midrule
         nOctaveLayers/Sublevels & $3$ \\
         contrastThreshold       & 0.04 \\
         edgeThreshold           & $10$ \\
         sigma                   & $1.6$ \\
         \bottomrule
      \end{tabular}
      \caption[Parameters used for SIFT]{Parameters used for SIFT (number of octaves, number of sublevels
      per octave, detector response threshold, edge threshold for filtering
   edges, standard deviation of the Gaussian for the initial image)}
      \label{tab:sift_params}
   \end{center}
\end{table}

With default parameters (\autoref{tab:sift_params}), SIFT compares much worse,
particularly on full resolution where it grossly overestimates the necessary
rotation. The results are better on the scaled-down images, possibly because of
the reduction of noise, but still only partly useful on the smallest resolution.

\begin{figure}
   \begin{subfigure}{.5\linewidth}
      \centering      
      \pgfplotstableread[col sep=space]{data/gut_rosenkrantz_detector_KAZE_resize_1_ratio_0.800000.dat}\datatable
\begin{tikzpicture}
   \begin{axis}[
         xlabel={Image Number},
         ylabel={Rotation angles},
         xtick=data,
         cycle list name=mylist,
         scale=.7,
      ]

      % original size
      \addplot table[
         x expr=\coordindex,
         y=realthetax,
      ] {\datatable};
      \addlegendentry{True $\theta_x$}
      \addplot table[
         x expr=\coordindex,
         y=realthetay,
      ] {\datatable};
      \addlegendentry{True $\theta_y$}
      \addplot table[
         x expr=\coordindex,
         y=realthetaz,
      ] {\datatable};
      \addlegendentry{True $\theta_z$}

      % half size
      \addplot table[
         x expr=\coordindex,
         y=thetax,
      ] {\datatable};
      \addlegendentry{$\theta_x$}
      \addplot table[
         x expr=\coordindex,
         y=thetay,
      ] {\datatable};
      \addlegendentry{$\theta_y$}
      \addplot table[
         x expr=\coordindex,
         y=thetaz,
      ] {\datatable};
      \addlegendentry{$\theta_z$}

   \end{axis}
\end{tikzpicture}



      \label{fig:manor_KAZE_rotation_1}
      \caption{$s=1$}
   \end{subfigure}
   \quad
   \begin{subfigure}{.5\linewidth}
      \centering      
      \pgfplotstableread[col sep=space]{data/gut_rosenkrantz_detector_KAZE_resize_2_ratio_0.800000.dat}\datatable
\begin{tikzpicture}
   \begin{axis}[
         xlabel={\vphantom{Image Number}},
         xtick=data,
         cycle list name=mylist,
         scale=.7,
         yticklabels={},
         ymin=-20,
         ymax=5,
      ]

      % original size
      \addplot table[
         x expr=\coordindex,
         y=realthetax,
      ] {\datatable};
      \addlegendentry{True $\theta_x$}
      \addplot table[
         x expr=\coordindex,
         y=realthetay,
      ] {\datatable};
      \addlegendentry{True $\theta_y$}
      \addplot table[
         x expr=\coordindex,
         y=realthetaz,
      ] {\datatable};
      \addlegendentry{True $\theta_z$}

      % half size
      \addplot table[
         x expr=\coordindex,
         y=thetax,
      ] {\datatable};
      \addlegendentry{$\theta_x$}
      \addplot table[
         x expr=\coordindex,
         y=thetay,
      ] {\datatable};
      \addlegendentry{$\theta_y$}
      \addplot table[
         x expr=\coordindex,
         y=thetaz,
      ] {\datatable};
      \addlegendentry{$\theta_z$}

   \end{axis}
\end{tikzpicture}




      \label{fig:manor_KAZE_rotation_2}
      \caption{$s=2$}
   \end{subfigure}\\[3ex]
   \begin{subfigure}{\linewidth}
      \centering      
      \pgfplotstableread[col sep=space]{data/gut_rosenkrantz_detector_KAZE_resize_4_ratio_0.800000.dat}\datatable
\begin{tikzpicture}
   \begin{axis}[
         scale=.7,
         cycle list name=mylist,
         xtick=data,
         ymin=-20,
         ymax=5,
      ]

      % original size
      \addplot table[
         x expr=\coordindex,
         y=realthetax,
      ] {\datatable};
      \addlegendentry{True $\theta_x$}
      \addplot table[
         x expr=\coordindex,
         y=realthetay,
      ] {\datatable};
      \addlegendentry{True $\theta_y$}
      \addplot table[
         x expr=\coordindex,
         y=realthetaz,
      ] {\datatable};
      \addlegendentry{True $\theta_z$}

      % half size
      \addplot table[
         x expr=\coordindex,
         y=thetax,
      ] {\datatable};
      \addlegendentry{$\theta_x$}
      \addplot table[
         x expr=\coordindex,
         y=thetay,
      ] {\datatable};
      \addlegendentry{$\theta_y$}
      \addplot table[
         x expr=\coordindex,
         y=thetaz,
      ] {\datatable};
      \addlegendentry{$\theta_z$}

   \end{axis}
\end{tikzpicture}





      \label{fig:manor_KAZE_rotation_4}
      \caption{$s=4$}
   \end{subfigure}
   \caption[Manor data: Rotation AKAZE]{Manor data set: Angles of rotation relative to reference with
   AKAZE features}
   \label{fig:manor_KAZE_rotation}
\end{figure}

\begin{figure}
   \begin{subfigure}{.5\linewidth}
      \centering      
      \pgfplotstableread[col sep=space]{data/gut_rosenkrantz_detector_SIFT_resize_1_ratio_0.800000.dat}\datatable
\begin{tikzpicture}
   \begin{axis}[
         xlabel={Image Number},
         ylabel={Rotation angles},
         xtick=data,
         cycle list name=mylist,
         scale=.7,
         ymin=-55,
         ymax=10,
      ]

      % original size
      \addplot table[
         x expr=\coordindex,
         y=realthetax,
      ] {\datatable};
      \addlegendentry{True $\theta_x$}
      \addplot table[
         x expr=\coordindex,
         y=realthetay,
      ] {\datatable};
      \addlegendentry{True $\theta_y$}
      \addplot table[
         x expr=\coordindex,
         y=realthetaz,
      ] {\datatable};
      \addlegendentry{True $\theta_z$}

      % half size
      \addplot table[
         x expr=\coordindex,
         y=thetax,
      ] {\datatable};
      \addlegendentry{$\theta_x$}
      \addplot table[
         x expr=\coordindex,
         y=thetay,
      ] {\datatable};
      \addlegendentry{$\theta_y$}
      \addplot table[
         x expr=\coordindex,
         y=thetaz,
      ] {\datatable};
      \addlegendentry{$\theta_z$}

   \end{axis}
\end{tikzpicture}



      \label{fig:manor_SIFT_rotation_1}
      \caption{$s=1$}
   \end{subfigure}
   \quad
   \begin{subfigure}{.5\linewidth}
      \centering      
      \pgfplotstableread[col sep=space]{data/gut_rosenkrantz_detector_SIFT_resize_2_ratio_0.800000.dat}\datatable
\begin{tikzpicture}
   \begin{axis}[
         xlabel={\vphantom{Image Number}},
         xtick=data,
         cycle list name=mylist,
         scale=.7,
         yticklabels={},
         ymin=-55,
         ymax=10,
      ]

      % original size
      \addplot table[
         x expr=\coordindex,
         y=realthetax,
      ] {\datatable};
      \addlegendentry{True $\theta_x$}
      \addplot table[
         x expr=\coordindex,
         y=realthetay,
      ] {\datatable};
      \addlegendentry{True $\theta_y$}
      \addplot table[
         x expr=\coordindex,
         y=realthetaz,
      ] {\datatable};
      \addlegendentry{True $\theta_z$}

      % half size
      \addplot table[
         x expr=\coordindex,
         y=thetax,
      ] {\datatable};
      \addlegendentry{$\theta_x$}
      \addplot table[
         x expr=\coordindex,
         y=thetay,
      ] {\datatable};
      \addlegendentry{$\theta_y$}
      \addplot table[
         x expr=\coordindex,
         y=thetaz,
      ] {\datatable};
      \addlegendentry{$\theta_z$}

   \end{axis}
\end{tikzpicture}




      \label{fig:manor_SIFT_rotation_2}
      \caption{$s=2$}
   \end{subfigure}\\[3ex]
   \begin{subfigure}{\linewidth}
      \centering      
      \pgfplotstableread[col sep=space]{data/gut_rosenkrantz_detector_SIFT_resize_4_ratio_0.800000.dat}\datatable
\begin{tikzpicture}
   \begin{axis}[
         scale=.7,
         cycle list name=mylist,
         xtick=data,
      ]

      % original size
      \addplot table[
         x expr=\coordindex,
         y=realthetax,
      ] {\datatable};
      \addlegendentry{True $\theta_x$}
      \addplot table[
         x expr=\coordindex,
         y=realthetay,
      ] {\datatable};
      \addlegendentry{True $\theta_y$}
      \addplot table[
         x expr=\coordindex,
         y=realthetaz,
      ] {\datatable};
      \addlegendentry{True $\theta_z$}

      % half size
      \addplot table[
         x expr=\coordindex,
         y=thetax,
      ] {\datatable};
      \addlegendentry{$\theta_x$}
      \addplot table[
         x expr=\coordindex,
         y=thetay,
      ] {\datatable};
      \addlegendentry{$\theta_y$}
      \addplot table[
         x expr=\coordindex,
         y=thetaz,
      ] {\datatable};
      \addlegendentry{$\theta_z$}

   \end{axis}
\end{tikzpicture}





      \label{fig:manor_SIFT_rotation_4}
      \caption{$s=4$}
   \end{subfigure}
   \caption[Manor data: Rotation SIFT]{Manor data set: Angles of rotation relative to reference with
   SIFT features}
   \label{fig:manor_SIFT_rotation}
\end{figure}

\subsection{Translation Estimation}

Lastly, the direction of necessary translation is evaluated in
\autoref{fig:manor_direction}. It is moot to discuss any improvement in
comparison with the train data set, as the results are also completely false,
SIFT displaying a larger variance than AKAZE, but neither are useful.

\begin{figure}[h]
   \begin{subfigure}{.5\linewidth}
      \centering      
      \pgfplotstableread[col sep=space]{data/gut_rosenkrantz_detector_KAZE_resize_1_ratio_0.800000.dat}\datatablenoresize
\pgfplotstableread[col sep=space]{data/gut_rosenkrantz_detector_KAZE_resize_2_ratio_0.800000.dat}\datatableresize
\pgfplotstableread[col
sep=space]{data/gut_rosenkrantz_detector_KAZE_resize_4_ratio_0.800000.dat}\datatabledoubleresize
\begin{tikzpicture}
   \begin{axis}[
         xlabel={Image Number},
         ylabel={Angular difference},
         xtick=data,
         legend pos=north east,
         scale=.6,
      ]
      \pgfplotstablegetrowsof{\datatablenoresize}
      \pgfmathsetmacro\yfin{\pgfmathresult - 6}
      \pgfmathsetmacro\yini{0}


      % original size
      \addplot table[
         skip coords between index={\yini}{\yfin},
         x expr=\coordindex,
         y expr={
            acos(
            (
            \thisrow{x}*\thisrow{realx} +
            \thisrow{y}*\thisrow{realy} +
            \thisrow{z}*\thisrow{realz}
            ) / 
            (
            sqrt(
            (\thisrow{x})^2 +
            (\thisrow{y})^2 +
            (\thisrow{z})^2
            ) * 
            sqrt(
            (\thisrow{realx})^2 +
            (\thisrow{realy})^2 +
            (\thisrow{realz})^2
            )
            )
            )
         }
      ] {\datatablenoresize};
      \addlegendentry{$s=1$}

      % quarter size
      \addplot table[
         skip coords between index={\yini}{\yfin},
         x expr=\coordindex,
         y expr={
            acos(
            (
            \thisrow{x}*\thisrow{realx} +
            \thisrow{y}*\thisrow{realy} +
            \thisrow{z}*\thisrow{realz}
            ) / 
            (
            sqrt(
            (\thisrow{x})^2 +
            (\thisrow{y})^2 +
            (\thisrow{z})^2
            ) * 
            sqrt(
            (\thisrow{realx})^2 +
            (\thisrow{realy})^2 +
            (\thisrow{realz})^2
            )
            )
            )
         }
      ] {\datatableresize};
      \addlegendentry{$s=2$}

      % sixteenth size
      \addplot table[
         skip coords between index={\yini}{\yfin},
         x expr=\coordindex,
         y expr={
            acos(
            (
            \thisrow{x}*\thisrow{realx} +
            \thisrow{y}*\thisrow{realy} +
            \thisrow{z}*\thisrow{realz}
            ) / 
            (
            sqrt(
            (\thisrow{x})^2 +
            (\thisrow{y})^2 +
            (\thisrow{z})^2
            ) * 
            sqrt(
            (\thisrow{realx})^2 +
            (\thisrow{realy})^2 +
            (\thisrow{realz})^2
            )
            )
            )
         }
      ] {\datatabledoubleresize};
      \addlegendentry{$s=4$}
   \end{axis}
\end{tikzpicture}

      \caption{AKAZE features}
      \label{fig:manor_KAZE_direction}
   \end{subfigure}
   \begin{subfigure}{.5\linewidth}
      \centering      
      \pgfplotstableread[col sep=space]{data/gut_rosenkrantz_detector_SIFT_resize_1_ratio_0.800000.dat}\datatablenoresize
\pgfplotstableread[col sep=space]{data/gut_rosenkrantz_detector_SIFT_resize_2_ratio_0.800000.dat}\datatableresize
\pgfplotstableread[col
sep=space]{data/gut_rosenkrantz_detector_SIFT_resize_4_ratio_0.800000.dat}\datatabledoubleresize
\begin{tikzpicture}
   \begin{axis}[
         xlabel=\vphantom{Image Number},
         % ylabel={Angular difference},
         yticklabel pos=right,
         xtick=data,
         legend pos=north east,
         scale=.6,
      ]
      \pgfplotstablegetrowsof{\datatablenoresize}
      \pgfmathsetmacro\yfin{\pgfmathresult - 6}
      \pgfmathsetmacro\yini{0}


      % original size
      \addplot table[
         skip coords between index={\yini}{\yfin},
         x expr=\coordindex,
         y expr={
            acos(
            (
            \thisrow{x}*\thisrow{realx} +
            \thisrow{y}*\thisrow{realy} +
            \thisrow{z}*\thisrow{realz}
            ) / 
            (
            sqrt(
            (\thisrow{x})^2 +
            (\thisrow{y})^2 +
            (\thisrow{z})^2
            ) * 
            sqrt(
            (\thisrow{realx})^2 +
            (\thisrow{realy})^2 +
            (\thisrow{realz})^2
            )
            )
            )
         }
      ] {\datatablenoresize};
      \addlegendentry{$s=1$}

      % quarter size
      \addplot table[
         skip coords between index={\yini}{\yfin},
         x expr=\coordindex,
         y expr={
            acos(
            (
            \thisrow{x}*\thisrow{realx} +
            \thisrow{y}*\thisrow{realy} +
            \thisrow{z}*\thisrow{realz}
            ) / 
            (
            sqrt(
            (\thisrow{x})^2 +
            (\thisrow{y})^2 +
            (\thisrow{z})^2
            ) * 
            sqrt(
            (\thisrow{realx})^2 +
            (\thisrow{realy})^2 +
            (\thisrow{realz})^2
            )
            )
            )
         }
      ] {\datatableresize};
      \addlegendentry{$s=2$}

      % sixteenth size
      \addplot table[
         skip coords between index={\yini}{\yfin},
         x expr=\coordindex,
         y expr={
            acos(
            (
            \thisrow{x}*\thisrow{realx} +
            \thisrow{y}*\thisrow{realy} +
            \thisrow{z}*\thisrow{realz}
            ) / 
            (
            sqrt(
            (\thisrow{x})^2 +
            (\thisrow{y})^2 +
            (\thisrow{z})^2
            ) * 
            sqrt(
            (\thisrow{realx})^2 +
            (\thisrow{realy})^2 +
            (\thisrow{realz})^2
            )
            )
            )
         }
      ] {\datatabledoubleresize};
      \addlegendentry{$s=4$}
   \end{axis}
\end{tikzpicture}


      \caption{SIFT features}
      \label{fig:manor_SIFT_direction}
   \end{subfigure}
   \caption[Manor data: Translation]{Manor data set: Angular difference between actually necessary translation and estimate}
   \label{fig:manor_direction}
\end{figure}

\FloatBarrier

\section{Summary}

Of the three pieces of information needed for user guidance, only the most
important one---the direction of translation---cannot be recovered to any
satisfying degree with this method. Both scale and necessary camera
rotation estimation work, at least if the movement over iterations is mostly
horizontal and not along the optical axis. A principal problem with the
estimation of necessary translation could be observed. If the movement is mostly
in one direction and the translation between reference and current frame is
computed as in \eqref{eq:necessary_trans}
\begin{equation*}
   -\sub{R}{ref,first}\sub{R^T}{current,first}\sub{T}{current,first} + \sub{T}{ref,first}
\end{equation*}
then both summands will have mostly the same orientation which will consequently
be zeroed out by the sign inversion of the first one. If the resulting vector is
then normalised to unit length, the other two dimensions will have nonzero
values determined by small differences in their orientation and thus by noise
and point into a completely wrong direction. Applying scale factors before the
addition did not improve the situation. It is evident that this solution cannot
work for movement in only on principal direction.

It could also be demonstrated that AKAZE features yield the more accurate
results, except on the smallest scale, where SIFT compares somewhat favourably.
The smallest scale however also leads to general deterioration in quality,
suggesting that a scale factor between $2$ and $4$ may be required to combine accuracy
with speed of processing.

Improvements for all estimates could possibly be improved by fine-tuning the
parameters of both descriptors to adapt them to scenes with buildings, which has
not been tried here.
