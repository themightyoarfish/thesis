\chapter{Evaluation}

The approach has been evaluated on two realistic datasets. The most important
questions are whether the direction of the necessary translation is correctly
identified and its scale decreasing with distance to the target.
For both sets of images, the ground-truth translation between each image and the
first frame has been measured with centimetre accuracy, while the ground-truth
rotation has been estimated from manually labeled correspondence as it is
difficult to measure without the proper instruments. Since for the case of
noise-free correspondences in a non-degenerate configuration, relative pose
estimation algorithms are mathematically correct, this has been deemed
sufficiently accurate to evaluate the procedure. For each image pair, 19--27
correspondences have been labeled, of which the majority is used for pose
recovery. For pose recovery, RANSAC is used in conjunction with the five-point
solver, a point is considered an inlier for a given essential matrix if its
distance to its epipolar line is no more than three pixels. These parameters
lead to the majority of points being inliers of the pose recovery, the few
outliers can be explained by imprecise labeling.

In both data sets, the translation was mostly in the horizontal direction and
along the optical axis; the vertical translation is thus neglected. Similarly,
rotation was applied mainly around the vertical axis.

In order to idealise the condition, the reference photograph has been used to
fill the role of the second frame for world scale computation. In reality, since
the reference location is unknown, the reference world scale is obtained from a
position somewhat off.

\section{Train Station Data Set}

A schematic bird's eye view of the captures is shown in \autoref{fig:train_data_scenario}.

\begin{figure}
   {\centering      
      \begin{tikzpicture}
   \begin{axis}[
         xmin = -1,
         xmax = 6,
         ymin = -1,
         ymax = 6,
         xlabel = $x$,
         ylabel = $y$,
         every node near coord/.append style={yshift=-0.5cm,anchor=-10} % hacky way putting nodes below points instead of above
      ]
      \addplot+[
         nodes near coords,
         only marks,
         point meta=explicit symbolic,
      ]
      coordinates {
         (0,0)        [0]
         (5.65,2.65)  [1]
         (3.12,0.16)  [2]
         (0.83,0.03)  [3]
         (0.20,0.03)  [4]
         (2,5)        [station]
      };

      \coordinate (station) at (axis cs:2,5);

      \foreach \x / \y in {0/0,5.65/2.65,3.12/0.16,0.83/0.03,0.20/0.03}
      {
         \edef\temp{
            \noexpand\node[outer sep=0,inner sep=0] (foo) at (axis cs:\x,\y) {}; 
         \noexpand\draw[->] (foo) -- ($ (foo) !1cm! (station) $);
      }
         \temp
      }

   \end{axis}
\end{tikzpicture}

   \caption{Schematic representation of the train data set. Lengths and angles are not
   precise.}
   \label{fig:train_data_scenario}}
\end{figure}
\autoref{tab:train_data} summarises the ground truth for the five images.

\subsection{Scale Estimation}

\autoref{fig:train_dist_ratio} shows how the average distance of points to the first
frame's camera varies with the second image used for triangulation. The plot
illustrates that---especially at full resolution---the ratio based on feature
matching closely mirrors the real value. The difference increases with
decreasing image scale, but the the slope of the graphs is quite similar. This
shows that indeed with increasing distance to the first frame, the ratio
decreases, allowing a deduction as to how close the camera is to the target, at
least with respect to previous iterations. The situation is somewhat worse at
the smallest scale with the ratio between the fifth third and fourth images
barely decreasing.

\newcolumntype{m}{>{$}c<{$}}

\begin{table}
  \caption{Ground truth for the train data. Image 0 is the reference frame,
  translations and rotations are given as in \eqref{eq:camera_transform}.}
  \begin{tabular}{cmmm}
     \toprule
     Image        & \text{Translation to reference} & \text{Rotation to reference} & \text{ratio}\\
     number       & [x,y,z]                         & [\theta_x, \theta_y, \theta_z]
     \\
     \midrule
     0 & [0      , 0 , 0]      & [0        , 0       , 0]       & 0\\
     1 & [ .9053 , 0 , .4246 ] & [ -3.3779 , -9.3779 , 1.05121] & 3.8936\\
     2 & [ .9986 , 0 , .0512 ] & [ -1.3274 , -5.7134 , -.1884 ] & 1.6461\\
     3 & [ .9993 , 0 , .0361 ] & [ -1.7156 , -2.4761 , .3469  ] & 1.0965\\
     4 & [ .9950 , 0 , .0995 ] & [ .054606 , -4.4867 , .2452  ] & 1.0343\\
  \end{tabular}
  \label{tab:train_data}
\end{table}

\begin{figure}
   {\centering      
      \input{gfx/train_dist_ratio_plot.tex}
   \caption{Evolution of the distance ratio between images in the Train Station
   data set}
   \label{fig:train_dist_ratio}}
\end{figure}

\section{Manor Data Set}

The schematic positions for the seven manor captures (including reference
photograph) are shown in \autoref{fig:manor_data_scenario}. In contrast to the
train station, there is prominent movement along the optical axis.

\begin{figure}
   {\centering      
      \begin{tikzpicture}
   \begin{axis}[
         xmin = -2,
         xmax = 20,
         ymin = -10,
         ymax = 16,
         xlabel = $x$,
         ylabel = $y$,
         every node near coord/.append style={anchor=5} % hacky way putting nodes below points instead of above
      ]
      \addplot+[
         nodes near coords,
         only marks,
         point meta=explicit symbolic,
      ]
      coordinates {
         (0,0)   [0]
         (2,0)   [1]
         (4,2)   [2]
         (4,4)   [3]
         (8,-5)  [4]
         (10,4)  [5]
         (16,-6) [6]
         (10,15) [manor]
      };

      \coordinate (manor) at (axis cs:10,15);

      \foreach \x / \y in {0/0,2/0,4/2,4/4,8/-5,10/4,16/-6,10/15}
      {
         \edef\temp{
            \noexpand\node[outer sep=0,inner sep=0] (foo) at (axis cs:\x,\y) {}; 
            \noexpand\draw[->] (foo) -- ($ (foo) !.5cm! (manor) $);
         }
         \temp
      }

   \end{axis}
\end{tikzpicture}

   \caption{Schematic representation of the manor data set. Lengths and angles are not
   precise.}
   \label{fig:manor_data_scenario}}
\end{figure}


