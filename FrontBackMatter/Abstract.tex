\chapter*{Abstract}

Rephotography is the process of recreating a historic photograph by finding the
exact pose and ideally the exact camera parameters to then take a picture from
the same spot. The original and new images
can be used to document the passage of time and the changes which a static scene
has undergone, for instance by blending the two images together. Traditionally,
the exercise is carried out by photographers via careful examination of the
current camera picture and comparing it with the original image, gradually
moving the camera until an optimal registration is achieved. Besides being very
laborious, this approach is also quite error-prone, motivating the desire for
computerised assistance.

The ubiquity of camera-enabled mobile devices which---contrarily to digital
cameras---can be programmed allows such assistance to be provided, but few aids
are available. Two existing mobile applications simplify the procedure, yet still the
photographer is required to determine the necessary motion on their own. This
thesis presents an attempt to reproduce a more sophisticated system which was
prototyped for a laptop with connected camera as a mobile application. This
approach makes use of image processing in order to tell the user how to move the
camera to recover the original viewpoint.

The theoretical and practical challenges in computing a necessary motion are
explored and the system implemented as an iOS application. A detailed
evaluation of the results is performed, concluding that the reproduction was
partially successful, but some aspects of the pose recovery require further work.
