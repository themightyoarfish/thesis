\chapter*{Abstract}

Rephotography is the process of recreating a historic photograph by finding the
exact pose and ideally the exact camera parameters. The original and new images
can be used to document the passage of time and the changes which a static scene
has undergone, for instance by blending the two images together. Traditionally,
the exercise is carried out by photographers by careful examination of the
current camera picture and comparing it with the original image, gradually
moving the camera until an optimal registration is achieved. Besides being very
labourous, this approach is also quite error-prone, motivating the desire for
computerised assistance.

The ubiquity of camera-enabled mobile devices which---contrarily to
cameras---can be programmed allows such assistance to be provided, but few aids
are available. Two mobile applications simplify the procedure, yet still the
photographer is required to determine the necessary motion on their own. This
thesis presents an attempt to reproduce a more sophisticated system which makes
use of image processing in order to tell the user how to move the camera to
recover the original viewpoint.

The theoretical and practical challenges in computing a necessary motion are
explored and the system implemented as an iOS application. A detailed
evaluation of the results is performed, concluding that the reproduction was
unsuccessful and requires further work.
