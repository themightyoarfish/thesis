\chapter*{Zusammenfassung}

\begin{otherlanguage}{german}

   Refotografie bezeichnet das Wiederfinden von Aufnahmepose und Kameraparametern
   einer möglicherweise historischen Fotografie, um eine Aufnahme vom selben
   Standort aus zu machen.  Das Original und das neue Bild
   können verwendet werden, um Veränderungen einer Szene über einen längeren
   Zeitraum zu dokumentieren, indem die Bilder beispielsweise übereinander gelegt
   oder Teile des einen in das andere Bild geblendet werden. Normalerweise wird die
   Prozedur von Fotografinnen mittels Ansicht eines Ausdrucks vom Originalbild und
   viel Geduld durchgeführt. Die Kamera wird hierbei so lange graduell bewegt, bis
   die aktuelle Einstellung möglichst genau der des Originals entspricht. Der
   visuelle Vergleich zwischen Vorlage und Kamerabild ist zeitaufwendig und
   fehleranfällig, was den Wunsch nach Computerunterstützung motiviert.

   Die Verbreitung von mobilen Geräten mit integrierten Kamera, die anders als
   kommerzielle Digitalkameras programmierbar sind, erlaubt solche Unterstützung.
   Bisher existieren allerdings kaum Ansätze. Zwei mobile Anwendungen vereinfachen 
   das Refotografieren mittels eines Overlays des Originalbildes über das
   Kamerabild, doch die Nutzerin muss die nötige Kamerabewegung nach wie vor selbst
   schätzen. Diese Arbeit stellt einen Versuch vor, ein leistungsfähigeres System
   zu untersuchen und zu implementieren, welches zuvor für
   einen Computer mit angeschlossener Kamera entwickelt wurde. Die Anwendung nutzt
   Algorithmen aus Bildverarbeitung und maschinellem Sehen, um der Nutzerin die
   nötige Bewegung zu kommunizieren.

   Die theoretischen und praktischen Herausforderungen bei der Berechnung der
   nötigen Bewegung werden untersucht und das System für iOS implementiert. Eine
   detaillierte Evaluation der Ergebnisse zeigt, dass die Reproduktion teilweise
   erfolgreich ist, wobei einige Aspekte beim Wiederfinden des
   Originalaufnahmeortes weiterer Arbeit bedürfen.

\end{otherlanguage}
