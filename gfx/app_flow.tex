\newcommand{\treenodedist}{1.8}
\begin{tikzpicture}[
      every node/.style={font=\ttfamily,draw=black,thick,anchor=west},
      grow via three points={
         one child at (0.5,-\treenodedist) and
      two children at (0.5,-\treenodedist) and (0.5,-2*\treenodedist)},
      edge from parent path={(\tikzparentnode.south) |- (\tikzchildnode.west)},
      edge from parent/.style={draw,-latex},
   ]

   \makegrayinprint % it makes no fucking sense that this is ignored when colours are set at begin of the tikzpicture

   \node {
      \begin{tabular}{l}
         MainViewController \\
         \rmfamily\small View or make rephotos
      \end{tabular}
   }
   child { node {
         \begin{tabular}{l}
            PhotoChooserController \\
            \rmfamily\small Select original or set settings
         \end{tabular}
      }
      child { node {
            \begin{tabular}{c}
               RephotoManager \\
               \rmfamily\small Retrieve viewpoint
            \end{tabular}
         }
         child { node {
               \begin{tabular}{c}
                  ResultViewController \\
                  \rmfamily\small Review rephoto
               \end{tabular}
            }
            child [missing] {}
         }
      }
      child [missing] {}
      child { node {
            \begin{tabular}{c}
               SettingsViewController \\
               \rmfamily\small Adjust settings
            \end{tabular}
         } 
      }
   }    
   child [missing] {}
   child [missing] {}
   child [missing] {}
   child { node {
         \begin{tabular}{c}
            ELCImagePickerController \\ 
            \rmfamily\small Select rephoto to view
         \end{tabular}
      }
      child { node {
            \begin{tabular}{c}
               ResultViewController \\
               \rmfamily\small Review rephoto
            \end{tabular}
         } 
      }
   };
\end{tikzpicture}
